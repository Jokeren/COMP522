\newcounter{alg:non-fifo:lines}
\begin{algo}[!ht]
\caption{SALSA implementation of SCPool: Data Structures.} 
\label{alg:non-fifo-ds}
\scriptsize
\begin{minipage}[t]{0.48\textwidth}
\begin{distribalgo}[1]
\smallskip

\INDENT {{\bf Chunk type}}
	\STATE Task[CHUNK\_SIZE] tasks 
  \STATE int owner \comment {owner's consumer id}
\ENDINDENT

\INDENT {{\bf Node type}}
  \STATE Chunk c; initially $\bot$
  \STATE int idx; initially -1
  \STATE Node next; 
\ENDINDENT

\setcounter{alg:non-fifo:lines}{\value{ALC@line}} % store the line number
\end{distribalgo}
\end{minipage}%
%
\hfill
%
\begin{minipage}[t]{0.48\textwidth}
%
\begin{distribalgo}[1]
\setcounter{ALC@line}{\value{alg:non-fifo:lines}}
\smallskip

\INDENT {{\bf SALSA per consumer data structure}:}
  \STATE int consumerId
  \STATE List\tup{Node}[] chunkLists \comment {one list per producer + extra list for stealing (every list is single-writer multi-reader)} 
  \STATE Queue\tup{Chunk} chunkPool \comment {pool of spare chunks}
  \STATE Node currentNode, initially $\bot$ \comment {current node to work with} 
\ENDINDENT

\setcounter{alg:non-fifo:lines}{\value{ALC@line}}
\end{distribalgo}
\end{minipage}
\end{algo}


\begin{figure}[htb]
	\centering
	\includegraphics[height=0.3\textwidth]{figures/salsa-struct}
	\caption{
	    \footnotesize{Chunk lists in SALSA single consumer pool implementation. Tasks are kept in chunks, which are 
	    organized in per-producer lists; an additional list is reserved for stealing. Each list can be modified 
	    by the corresponding producer only. The only process that is allowed to retrieve tasks from a chunk is 
	    the owner of that chunk (defined by the ownership flag). A Node's index corresponds to the latest task taken from the chunk
	    or the task that is about to be taken by the current chunk owner. 
	    }}
	\label{fig:salsa-struct}
\end{figure}

The SALSA data structure of a consumer $c_i$ is described in Algorithm~\ref{alg:non-fifo-ds} and partially depicted in Figure~\ref{fig:salsa-struct}. 
%We describe the implementation of a SALSA SCPool of some consumer $c_i$.
%Its data structures are described in Algorithm~\ref{alg:non-fifo-ds} and partially depicted in Figure~\ref{fig:salsa-struct}. 
The tasks inserted to SALSA are kept in chunks, which are organized in per-producer chunk lists. Only the producer mapped to a given list can insert a task to any chunk in that list. Every chunk is owned by a single consumer whose id is kept in the \emph{owner} field of the chunk.
The owner is the only process that is allowed to take tasks from the chunk; if another process wants to take a task from the chunk, it should first steal the chunk and change its ownership. The owner of a chunk
residing at $c_i$'s SCPool is $c_i$ itself, unless that chunk is being stolen. A task entry in a chunk is used at most once. It holds the value $\bot$ initially, and TAKEN after the task it held has been consumed.

The per-producer chunk lists are kept in the array \emph{chunkLists} (see Figure~\ref{fig:salsa-struct}), where \emph{chunkLists[j]} keeps a list of chunks with tasks inserted by producer $p_j$. In addition, the array has a special entry \emph{chunkLists[steal]}, holding chunks stolen by $c_i$. Every list has a single writer who can modify the list structure (add or remove nodes), \emph{chunkLists[j]}'s modifier is the producer $p_j$, while \emph{chunkLists[steal]}'s modifer is the SCPool's owner. For this purpose, a consumed node (whose chunk was removed by the consumer) is lazily reclaimed and removed by the list's owner. For brevity, we omit the linked list manipulation functions from the pseudo-code bellow. Allowing only one writer makes it possible to the lists without synchronization primitives, similarly to the single-writer linked-list in~\cite{Michael:2004:HPS:987524.987595}.

In addition to the chunk pointer, each node keeps the index of the latest taken task in the chunk. As we show in Section~\ref{alg-stealing}, this index plays a crucial role in chunk stealing. Safe memory reclamation is provided by using hazard pointers~\cite{Michael:2004:HPS:987524.987595} both for nodes and for chunks.

The free (reclaimed) chunks in SALSA are kept inside the SCPools in \emph{chunkPool}, a lock-free queue~\cite{Michael:1996:SFP:248052.248106}. These pools serve two purposes. First, they enable efficient memory reuse. Second, as we show in Section~\ref{alg-pools}, managing per-consumer chunk pools is good for load balancing. 	