%As modern computing hardware constantly grows in the number of cores,
%software written for emerging architectures must be highly scalable.
%At the same time, memory access has become the principle bottleneck,
%and new systems employ \emph{NUMA architectures}, where memory
%access from different chips is asymmetric.
%From the software's perspective, this increases the importance of
%cache-friendliness, reduced contention on shared memory locations,
%especially across chips, and local data access.
%As systems get larger, their behavior becomes less predictable, and
%so robustness to varying loads and unexpected thread stalls is
%essential.
%
%One of the fundamental building blocks of highly parallel software is
%a producer-consumer task pool. This paper presents a scalable and highly-efficient
%non-blocking pool, with lightweight and synchronization free operations on the fast-path.
%Its data allocation scheme is cache-friendly and highly suitable for NUMA environments. 
%Moreover, our pool is robust in the face of imbalanced loads and unexpected thread stalls.

Emerging computer architectures pose many new challenges for software
development. First, as the number of computing elements constantly
increases, the importance of \emph{scalability} of parallel programs
becomes paramount. Second, accessing memory has become the principal
bottleneck, while multi-CPU systems are based on NUMA architectures,
where memory access from different chips is asymmetric.
Therefore, it is instrumental to design software with
\emph{local data access}, \emph{cache-friendliness}, and
\emph{reduced contention} on shared memory locations, especially
across chips. Furthermore, as systems get larger, their behavior
becomes less predictable, underscoring the importance of \emph{robust}
programs that can overcome unexpected thread stalls.

Our overarching goal is to devise a methodology for developing
parallel algorithms addressing these challenges. In this paper,
we focus on one of the fundamental building blocks of highly parallel software, namely a producer-consumer task pool. Specifically, we
present a scalable and highly-efficient non-blocking pool, with
lightweight synchronization-free operations in the common case. Its
data allocation scheme is cache-friendly and highly suitable for NUMA
environments. Moreover, our pool is robust in the face of imbalanced
loads and unexpected thread stalls.

%%prod-cons pools are common, 
%The producer-consumer pool is a fundamental data structure consisting of an unordered collection of objects. Pools have a number of important
%applications in multiprocessor computing, e.g., transferring tasks in a parallel computation. 
%It is thus highly important to ensure that such a pool does not become a bottleneck when concurrently accessed 
%by large number of threads. 
%
%%fifo is not needed: on the one hand it dofeks performance, on the other hand it is hardly usable (talk about the case with multiple consumers)
%One of the common approaches to implement a producer-consumer pool is using FIFO or LIFO queues. 
%However, this approach inherently suffers from poor scalability and high synchronization costs~\cite{Afek:2010:SPP:1885276.1885295,Basin:2011:CST:2075029.2075087,Sundell:2011:LAC:1989493.1989550}. 
%In addition, FIFO/LIFO properties of the queues \emph{cannot be used in practice} if multiple consumers
%work on the same queue simultaneously.
%This happens because every consumer can be suspended by the OS scheduler for an unbounded period of time after retrieving a task.
%This way, a task can be ``bypassed'' by an arbitrary number of later tasks before actually being consumed. 
%Hence, even if a multi-consumer queue guarantees an order on task retrieval, no simple way exists to exploit such an order. 
%
%%we want to achieve scalable and highly efficient prod-cons pool (no need for FIFO as state above)
%%the desired properties of the pool implementation are as follows:
%% - locality conscious (cache-friendly for each separate thread)
%% - NUMA awareness (each thread works with closest memory)
%% - low contention among threads (threads work in different memory areas)
%% - infrequent synchronization operations
%This paper presents a scalable and highly-efficient non-blocking producer-consumer pool that does not guarantee any order on task retrieval.
%Its operations in the common case are lightweight and synchronization free, the tasks are kept in a cache-friendly manner and its data allocation schemes are highly suitable for NUMA environments. 

%Our system design follows the following principles: 
%1) locality awareness: our algorithm is cache-friendly; 2) NUMA awareness: in the common case each thread works with the memory local to its processor; 3) low contention: in the common case consumers and producers access disjoint memory regions; 4) lightweight fast path: most of the time threads do not invoke strong atomic operations and have low step complexity.
%We now present the techniques used for achieving these principles.

%separate mechanism from policy, 
%short description of what are the building blocks (per-consumer pools)
Our system is composed of two independent logical entities: 1) \emph{SALSA, Scalable and Low Synchronization Algorithm}, a single-consumer pool that exports a stealing operation, and 2) a work stealing framework implementing a management policy that operates multiple SALSA pools. 

% both produce and consume are lightweight
In order to improve locality and facilitate stealing, SALSA keeps tasks in chunks, organized in per-producer chunk lists. Only the producer mapped to a given list can insert
tasks to chunks in this list, which eliminates the need for synchronization among producers. 

Though each consumer has its own task pool, inter-consumer synchronization is required in order to allow stealing. The challenge is to do so without resorting to costly atomic operations (such as CAS or memory fences) upon each task retrieval. We address this challenge via a novel chunk-based stealing algorithm that allows consume operations to be synchronization-free in the common case, when no stealing occurs, which we call the \emph{fast path}. 
Moreover, SALSA reduces the stealing rate by moving entire chunks of tasks in one steal operation, which requires only two CAS operations. 

% NUMA awareness
In order to achieve locality of memory access on a NUMA architecture, SALSA chunks are kept in the consumer's local memory.
The management policy matches producers and consumers according to their proximity, 
which allows most task transfers to occur within a NUMA node.

% chunk pool producer-based balancing 
In many-core machines running multiple applications, system behavior
becomes less predictable. Unexpected thread stalls may lead
to an asymmetric load on consumers, which may in turn lead to high
stealing rates, hampering performance. SALSA employs a novel
auto-balancing mechanism that has producers insert tasks to
less loaded consumers, and is thus robust to spurious load
fluctuations.
%In many-core machines running multiple applications, system behavior becomes less predictable.
%Various fluctuations, e.g., stalled threads, may lead to asymmetric load on consumers, which increases the stealing rate and thus degrades the performance. 
%In SALSA, we have producers auto-balance the load by inserting tasks to less loaded consumers, and therefore decrease the need for chunk stealing.

We have implemented SALSA in C++, and tested its performance on a $32$-core NUMA machine. Our experiments show that the SALSA-based work stealing pool \emph{scales linearly} with the number of threads; it is $20$ times faster than other work-stealing alternatives, and shows more than a five-fold improvement over state-of-the-art non-FIFO alternatives. In addition, SALSA-based pools are robust to temporary stalls of participating threads and scale well even in unbalanced scenarios. 

This paper proceeds as follows. Section~\ref{sec:related} describes related work. We give the system overview in Section~\ref{sec:system}. The SALSA single-consumer algorithm is described in Section~\ref{sec:algo} and its correctness is discussed in Section~\ref{sec:correctness}. We discuss our experimental results in Section~\ref{sec:evaluation}, and finally conclude in Section~\ref{sec:conclusions}.