%prod-cons pools are common, 
The producer-consumer pool is a fundamental data structure consisting of an unordered collection of objects. Pools have a number of important
applications in multiprocessor computing, e.g., transferring tasks in a parallel computation. 
It is thus highly important to ensure that such a pool does not become a bottleneck when it is concurrently accessed 
by a large number of threads. 

%fifo is not needed: on the one hand it dofeks performance, on the other hand it is hardly usable (talk about the case with multiple consumers)
One of the common approaches to implement a producer-consumer pool is using FIFO or LIFO queues. 
However, this approach inherently suffers from poor scalability and high synchronization cost~\cite{Afek:2010:SPP:1885276.1885295,Basin:2011:CST:2075029.2075087,Sundell:2011:LAC:1989493.1989550}. 
In addition to that, the FIFO/LIFO properties of the queue \emph{cannot be used in practice} if multiple consumers
work on the same queue simultaneously.
This happens because every consumer can be suspended by the OS scheduler for an unbounded period of time right after retrieving a task.
This way, the first task can be ``bypassed'' by an arbitrary number of later tasks before actually being consumed. 
Hence, even if a multi-consumer queue guarantees an order on task retrieval, no simple way exists to exploit such an order. 

%we want to achieve scalable and highly efficient prod-cons pool (no need for FIFO as state above)
%the desired properties of the pool implementation are as follows:
% - locality conscious (cache-friendly for each separate thread)
% - NUMA awareness (each thread works with closest memory)
% - low contention among threads (threads work in different memory areas)
% - infrequent synchronization operations
This paper presents a scalable and highly-efficient non-blocking producer-consumer pool that does not guarantee any order on task retrieval.
Its operations in the common case are lightweight and synchronization free, the tasks are kept in a cache-friendly manner and its data allocation schemes are highly suitable for NUMA environments. 

%Our system design follows the following principles: 
%1) locality awareness: our algorithm is cache-friendly; 2) NUMA awareness: in the common case each thread works with the memory local to its processor; 3) low contention: in the common case consumers and producers access disjoint memory regions; 4) lightweight fast path: most of the time threads do not invoke strong atomic operations and have low step complexity.
%We now present the techniques used for achieving these principles.

%separate mechanism from policy, 
%short description of what are the building blocks (per-consumer pools)
Our system is composed of two independent logical entities: 1) \emph{SALSA, Scalable and Low Synchronization Algorithm}, a single-consumer pool that allows task stealing, and 2) a management policy that is responsible for operating multiple SALSA instances. 

% both produce and consume are lightweight
SALSA tasks are kept in chunks, which are organized in per-producer chunk lists. Only the producer mapped to a given list can insert
a task to any chunk in that list, which eliminates the need for synchronization among producers. In contrast, SALSA's consume operation must communicate with other consumers when they steal tasks and therefore cannot be fully synchronization free.
We note, however, that it would be wasteful and not efficient to invoke costly atomic operations upon each task retrieval just because of relatively rare stealing occurrences. 
Therefore, we propose a novel chunk-based stealing algorithm that allows consume operations to be lightweight and synchronization-free when no stealing happens. 

% NUMA awareness
In order to achieve the NUMA locality of memory accesses, SALSA chunks are allocated in the memory of the corresponding consumer's processor. 
The management policy then matches producers and consumers according to their proximity, which causes most of the task transfers to happen inside the same NUMA node. As steal operations might spoil NUMA locality, SALSA reduces their rate by moving the whole chunks of tasks upon each steal, in a way that requires two CAS operations only.  
% chunk pool producer-based balancing 

Another way to reduce the number of steal operations is to feed the consumers with respect to their load and consumption rates.
We introduce an elegant mechanism for maintaining dynamic per-consumer chunk pools, which enables producers to detect overloaded consumers. This way, producers auto-balance the load by inserting the tasks to the less loaded consumers and thus decrease the need for chunk stealing.

We have implemented SALSA in C++, and tested its performance on a $32$-cores machine. Our experiments demonstrate that the SALSA-based work stealing pool \emph{scales linearly} with the number of threads -- it is $\times20$ faster than other work-stealing alternatives and more than $\times5$ faster than the non-FIFO Concurrent Bags algorithm by Sundell et. al~\cite{Sundell:2011:LAC:1989493.1989550}. In addition, SALSA-based pools are robust to temporary stalls of participating threads and they scale well even in unbalanced scenarios with excessive number of steals. 

This paper proceeds as follows. Section~\ref{sec:related} describes related work. We give the system overview in Section~\ref{sec:system}. SALSA algorithm is described in Section~\ref{sec:algo} and its correctness is shown in Section~\ref{sec:correctness}. We discuss our experimental results in Section~\ref{sec:evaluation}, and finally conclude in Section~\ref{sec:conclusions}.