\documentclass{sig-alternate}

\setlength{\pdfpagewidth}{8.5in}
\setlength{\pdfpageheight}{11in}

\pdfoptionpdfminorversion=5

%\usepackage{times}
%\usepackage{fullpage}
%\usepackage{epic}
\usepackage{subfigure}
%\usepackage{wrapfig}
%\usepackage{eepic}
%\usepackage{epsfig}
\usepackage{color}
%\usepackage{amsmath,amsthm,amssymb}
%\usepackage{amsfonts}
\usepackage{float}
%\usepackage{appendix}
%\usepackage{multirow}
%\usepackage{booktabs}
\usepackage{url}

\renewcommand\floatpagefraction{1.0}
\renewcommand\topfraction{1.0}
\renewcommand\bottomfraction{0.9}
\renewcommand\bottomfraction{1.0}
\renewcommand\textfraction{0.0}

\newenvironment{enum*}%
 {\begin{enumerate}%
   %\setlength{\itemsep}{-5pt}%
   \setlength{\parsep}{-5pt}%
   \setlength{\topsep}{-5pt}}%
 {\end{enumerate}}

\newenvironment{item*}%
 {\begin{itemize}%
   %\setlength{\itemsep}{-5pt}%
   \setlength{\parsep}{-5pt}%
   \setlength{\topsep}{-5pt}}%
 {\end{itemize}}

\newcommand{\tup}[1]{%
        \relax\ifmmode
	           \langle #1 \rangle%
        \else
                $\langle$#1$\rangle$%
        \fi
}

%\theoremstyle{plain}
\floatstyle{ruled}
\newfloat{algo}{htbp}{algo}
\floatname{algo}{Algorithm}
\usepackage[noend]{algorithmic}
\usepackage{distribalgo}

\newtheorem{thm}{Theorem}
\newtheorem{lemma}{Lemma}
\newtheorem{claim}{Claim}
\newtheorem{corollary}{Corollary}
\newtheorem{definition}{Definition}
\newtheorem{property}{Property}
\newtheorem{proposition}{Proposition}
\newtheorem{observation}{Observation}
\newtheorem{conjecture}{Conjecture}
\newtheorem{designrule}{Design Principle}
\newtheorem{invariant}{Invariant}
\newtheorem{theorem}{Theorem}

\newcommand{\up}[1]{\ensuremath{^{\textrm{#1}}}}
\newcommand{\down}[1]{\ensuremath{_{\textrm{#1}}}}
\newcommand{\tb}{\makebox[0.6cm]{}}
\newcommand{\negspace}{\vspace{-0.6\baselineskip}}
\newcommand{\snegspace}{\vspace{-0.25\baselineskip}}
\def\hyph{-\penalty0\hskip0pt\relax}

\newenvironment{restate}[1]{\begin{trivlist} \item {\bf #1 (restated)} \em}
  {\end{trivlist}}

\definecolor{Gray}{rgb}{0.1,0.4,0.1}
\definecolor{DarkBlue}{rgb}{0.2,0.2,0.5}
\definecolor{Cyan}{rgb}{0.5,0.2,0.5}
\newcommand{\comment}[1]{{\color{Gray}{$\rhd$ #1}}}
\newcommand{\elcomment}[1]{\hfill{\comment{#1}}}
\newcommand{\funcname}[1]{\textsc{\color{DarkBlue}{#1}}}
\newcommand{\typename}[1]{\textbf{\color{Cyan}{#1}}}
%
%\setlength\topmargin{-0.025in}
%\setlength\textheight{8.75in}

\begin{document}

\conferenceinfo{SPAA'12,} {June 25--27, 2012, Pittsburgh, Pennsylvania, USA.} 
\CopyrightYear{2012} 
\crdata{978-1-4503-1213-4/12/06} 
\clubpenalty=10000 
\widowpenalty = 10000

\title{SALSA: Scalable and Low Synchronization NUMA-aware Algorithm for Producer-Consumer Pools}

\numberofauthors{4}

\author{
\alignauthor
Elad Gidron\\
\affaddr{CS Department}\\
\affaddr{Technion, Haifa, Israel}\\
\email{eladgi@cs.technion.ac.il}
\alignauthor
Idit Keidar\\
\affaddr{EE Department}\\
\affaddr{Technion, Haifa, Israel}\\
\email{idish@ee.technion.ac.il}
\and
\alignauthor
Dmitri Perelman\titlenote{This work was partially supported by Hasso Plattner Institute.}\\
\affaddr{EE Department}\\
\affaddr{Technion, Haifa, Israel}\\
\email{dima39@tx.technion.ac.il}
\alignauthor
Yonathan Perez\\
\affaddr{EE Department}\\
\affaddr{Technion, Haifa, Israel}\\
\email{yonathan0210@gmail.com}
}

\date{}

\maketitle 

\begin{abstract}
We present a highly-scalable non-blocking producer-consumer task pool, designed with a special emphasis on lightweight synchronization and data locality.
The core building block of our pool is \emph{SALSA, Scalable And Low Synchronization Algorithm} for a single-consumer container with task stealing support. Each consumer operates on its own SALSA container, stealing tasks from other containers if necessary. We implement an elegant self-tuning policy for task insertion, which does not push tasks to overloaded SALSA containers, thus decreasing the likelihood of stealing. 

SALSA manages large chunks of tasks, which improves locality and facilitates stealing. 
%The use of page-size chunks is perfectly suitable for data migration among processor nodes in NUMA architectures. 
SALSA uses a novel approach for coordination among consumers, without strong atomic operations or memory barriers in the fast path. It invokes only two CAS operations during a chunk steal. 

Our evaluation demonstrates that a pool built using SALSA containers scales \emph{linearly} with the number of threads and significantly outperforms other FIFO and non-FIFO alternatives.

\end{abstract}

% A category with the (minimum) three required fields
\category{D.1.3}{Software}{Concurrent Programming}

\terms{Algorithms, Performance}

\keywords{Multi-core, concurrent data structures}

\section{Introduction}
\label{sec:intro}
%prod-cons pools are common, 
The producer-consumer pool is a fundamental data structure consisting of an unordered collection of objects. Pools have a number of important
applications in multiprocessor computing, e.g., transferring tasks in a parallel computation. 
It is thus highly important to ensure that such a pool does not become a bottleneck when concurrently accessed 
by large number of threads. 

%fifo is not needed: on the one hand it dofeks performance, on the other hand it is hardly usable (talk about the case with multiple consumers)
One of the common approaches to implement a producer-consumer pool is using FIFO or LIFO queues. 
However, this approach inherently suffers from poor scalability and high synchronization costs~\cite{Afek:2010:SPP:1885276.1885295,Basin:2011:CST:2075029.2075087,Sundell:2011:LAC:1989493.1989550}. 
In addition, FIFO/LIFO properties of the queues \emph{cannot be used in practice} if multiple consumers
work on the same queue simultaneously.
This happens because every consumer can be suspended by the OS scheduler for an unbounded period of time after retrieving a task.
This way, a task can be ``bypassed'' by an arbitrary number of later tasks before actually being consumed. 
Hence, even if a multi-consumer queue guarantees an order on task retrieval, no simple way exists to exploit such an order. 

%we want to achieve scalable and highly efficient prod-cons pool (no need for FIFO as state above)
%the desired properties of the pool implementation are as follows:
% - locality conscious (cache-friendly for each separate thread)
% - NUMA awareness (each thread works with closest memory)
% - low contention among threads (threads work in different memory areas)
% - infrequent synchronization operations
This paper presents a scalable and highly-efficient non-blocking producer-consumer pool that does not guarantee any order on task retrieval.
Its operations in the common case are lightweight and synchronization free, the tasks are kept in a cache-friendly manner and its data allocation schemes are highly suitable for NUMA environments. 

%Our system design follows the following principles: 
%1) locality awareness: our algorithm is cache-friendly; 2) NUMA awareness: in the common case each thread works with the memory local to its processor; 3) low contention: in the common case consumers and producers access disjoint memory regions; 4) lightweight fast path: most of the time threads do not invoke strong atomic operations and have low step complexity.
%We now present the techniques used for achieving these principles.

%separate mechanism from policy, 
%short description of what are the building blocks (per-consumer pools)
Our system is composed of two independent logical entities: 1) \emph{SALSA, Scalable and Low Synchronization Algorithm}, a single-consumer pool that supports task stealing, and 2) a management policy that is responsible for operating multiple SALSA instances. 

% both produce and consume are lightweight
SALSA tasks are kept in chunks, which are organized in per-producer chunk lists. Only the producer mapped to a given list can insert
tasks to chunks in this list, which eliminates the need for synchronization among producers. In contrast, SALSA's consume operation must communicate with other consumers when they steal tasks, and therefore cannot be fully synchronization free.
However, it would be inefficient to use costly atomic operations upon each task retrieval just because of relatively rare steal operations. 
To this end, we propose a novel chunk-based stealing algorithm that allows consume operations to be lightweight and synchronization-free when no stealing occurs. 

% NUMA awareness
In order to achieve the NUMA locality of memory accesses, SALSA chunks are allocated in the memory of the consumer's processor. 
The management policy matches producers and consumers according to their proximity, which causes most of the task transfers to occur inside the same NUMA node. As steal operations might harm NUMA locality, SALSA reduces their rate by moving whole chunks of tasks when stealing, in a way that requires only two CAS operations.  
% chunk pool producer-based balancing 

Another way to reduce the number of steal operations is to feed the consumers with respect to their load and consumption rates.
We introduce an elegant mechanism for maintaining dynamic per-consumer chunk pools, which enables producers to detect overloaded consumers. This way, producers auto-balance the load by inserting tasks to less loaded consumers, and thus decreasing the need for chunk stealing.

We have implemented SALSA in C++, and tested its performance on a $32$-cores NUMA machine. Our experiments show that the SALSA-based work stealing pool \emph{scale linearly} with the number of threads -- it is $\times20$ faster than other work-stealing alternatives, and more than $\times5$ faster than the non-FIFO Concurrent Bags algorithm of Sundell et. al~\cite{Sundell:2011:LAC:1989493.1989550}. In addition, SALSA-based pools are robust to temporary stalls of participating threads and scale well even in unbalanced scenarios with excessive number of steals. 

This paper proceeds as follows. Section~\ref{sec:related} describes related work. We give the system overview in Section~\ref{sec:system}. The SALSA algorithm is described in Section~\ref{sec:algo} and its correctness is shown in Section~\ref{sec:correctness}. We discuss our experimental results in Section~\ref{sec:evaluation}, and finally conclude in Section~\ref{sec:conclusions}.

\section{Related Work}
\label{sec:related}
\paragraph{Task pools\\}
There is a large body of work on lock-free unbounded FIFO queues and LIFO
stacks\cite{Gidenstam:2010:CLQ:1940234.1940266,Hendler:2004:SLS:1007912.1007944,
Hoffman:2007:BQ:1782394.1782423, Michael:1996:SFP:248052.248106,Moir:2005:UEI:1073970.1074013},
The problem with such algorithms is that due to the inherent need for ordering of operations they
scale poorly and provide low locality, and are therefore not useful as a consumer-producer task
pool. 

A number of previous works have recognized this limitation of FIFO, and observed that strict FIFO
order is seldom needed in multi-core systems
\cite{Afek:2010:SPP:1885276.1885295,springerlink:10.1007/978-3-642-17653-1_29,
Basin:2011:CST:2075029.2075087,Sundell:2011:LAC:1989493.1989550}. 

The closest such system to our implementation is to our work is the Concurrent Bag of Sundell et al.
\cite{Sundell:2011:LAC:1989493.1989550}, a non-FIFO producer-consumer pool, which, like SALSA, is
composed of per-producer chunk lists. Unlike our pool however, their consume operation requires a
strong atomic operation and steals are performed in the granularity of single tasks and not whole
chunks. Moreover, those pools are not scalable (i.e., the in system throughput does not increase as
more threads are added).

Other pools provide similar properties, and are not scalable as well. And also use different
techniques then we do in our work.

\paragraph{Techniques\\}
Some of the techniques used in our work were previously used in other works in different
variations. Work-stealing is a standard way to reduce contention by using separate per-consumer
pools \cite{Blumofe:1999:SMC:324133.324234} where task may be stolen from one pool to another.
In the common case where no steals are made, the algorithm may ran a \emph{fast-path} with
no synchronization. The concept of a synchronization-free fast-path appears in scheduling queues
\cite{Arora:1998:TSM:277651.277678}. However, these works assumes the same process is both the
producer and consumer, and hence synchronization-free operations were only used for a process to
transfer data to \emph{itself}. In our pool on the other hand, there is a distinction between
producers and consumers, and moreover, multiple producer may work on a single consumer pool even
without stealing. We do not know of any other work that achieves a synchronization-free fast-path
among multiple producers as we do. Furthermore, our steal algorithm steals a whole chunk of tasks
rather than just one task as most works do. \cite{Hendler:2002:NSW:571825.571876} is the only
exception we are familiar with which allows steals of multiple items, however it handles only
static-sized arrays and uses synchronization operations in the fast-path. 

Chunk-based data-structures
\cite{Braginsky:2011:LLL:1946143.1946153, 
Gidenstam:2010:CLQ:1940234.1940266,
Hendler:2006:DNW:1160290.1160294, Sundell:2011:LAC:1989493.1989550} use linked-list of chunks so
that the data structure will be of dynamic size while maintaining data-locality. We also use
chunk-based lists for that reason, furthermore, in SALSA steals are done in resolution of chunks.

\section{System Overview}
\label{sec:system}
In the current section we present our framework for scalable and NUMA-aware producer-consumer data exchange. 
Our system follows the principle of separating mechanism and policy.
To this end, we consider two independent logical entities: 
\begin{enumerate}
	\item \emph{A single consumer pool (SCPool)} mechanism manages the tasks arriving to a given consumer while introducing the possibility of stealing some tasks by other consumers.
	\item A management policy is responsible for operating SCPools: the policy routes producers' requests to the appropriate consumers and initiates stealing between the pools. This way, the policy controls the system's behavior according to considerations of load-distribution, throughput, fairness, locality, etc.
\end{enumerate} 

\newcounter{alg:non-fifo:lines}
\begin{algo}[!ht]
\caption{API for a Single Consumer Pool with stealing support.} 
\label{alg:scpool-api}
\begin{distribalgo}[1]
%\footnotesize

%\INDENT {\bf SCPool API:}
	\STATE boolean: produce(Task, SCPool) \elcomment {Tries to insert the task to the pool, returns false if no space is available.}
 	\STATE void: produceForce(Task, SCPool) \elcomment {Insert the task to the pool, expanding the pool if necessary. }
	\STATE \{Task $\cup \bot$\}: consume() \elcomment {Retrieve a task from the pool, returns $\bot$ if no tasks in the pool are detected.}
	%\STATE getStealingScore() \elcomment {Returns a score corresponding to the amount of tasks to steal.}
	\STATE \{Task $\cup \bot$\}: steal(SCPool from) \elcomment{Try to steal a number of tasks from the given pool and move them to the current pool. Return some stolen task or $\bot$. }%We guarantee that if there are tasks in the \emph{from} pool at the beginning of steal invocation, then either steal function returns a task, or there is another thread that returns a task during the steal execution.}
%\ENDINDENT
\setcounter{alg:non-fifo:lines}{\value{ALC@line}}
\end{distribalgo}
\end{algo}


The SCPool API provides the abstraction of a single-consumer task pool with stealing support, see Algorithm~\ref{alg:scpool-api}.
A producer can invoke two types of insertion operations: \emph{produce}, which attempts to insert a task to the given pool and fails if the pool is full, and \emph{produceForce}, which always succeeds by expanding the pool on demand.
There are also two ways to retrieve a task from the pool: the owner of the pool (only) can call the \emph{consume} function; while any other thread can invoke the \emph{steal} function, which tries to transfer a number of tasks between the pools and to return one of the stolen tasks. 
The pool must guarantee the following \emph{stealing property}, which is necessary for system liveness:
\begin{property}
If a pool is not empty at the beginning of the steal operation, then either the steal operation retrieves a task, or another thread retrieves a task during the steal execution.
\end{property}

\begin{figure}[htb]
	\centering
	\includegraphics[width=0.7\textwidth]{figures/system-fig}
	\caption{\footnotesize{System overview of the management framework. A producer $p_i$ observes an access list of available consumers for task insertion ordered according to their distance from $p_i$. A consumer $c_i$ observes a steal list of available consumers for task stealing ordered according to their distance from $c_i$. }}
	\label{fig:system-fig}
\end{figure}

As mentioned earlier, a management policy operates the SCPools in a way that is appropriate for the given desired system properties.
It is generally defined by the way a producer inserts a task to one of the SCPools and a consumer retrieves a task from its own pool / steal tasks from other pools. 
Note that the policy is independent of the underlying SCPool implementation, we believe that it is a subject for engineering optimizations, based on specific workloads and demands.


In the current work, we present a policy that exploits the locality properties of NUMA architectures and is aimed at achieving maximal throughput. If the individual SCPools themselves are lock-free and starvation-free (in the sense that no task remains untaken forever), then our policy preserves these properties at the system level. 
\begin{itemize}
	\item {\bf Access order list.} Each process in the system (producer or consumer) is provided with an ordered list of consumers, sorted according to their distance from that process (see Figure~\ref{fig:system-fig}). Intuitively, we build a system in which a producer mostly interacts with the closest consumer, and stealing mainly happens inside the same processor node. 
	\item {\bf Producer's policy.} A producer inserting a task first calls the \emph{produce} function of the first SCPool in the list. Note that a produce operation might fail if the pool is full, (which can be seen as evidence of that the corresponding consumer is overloaded).  In this case, the producer tries to insert the task into other pools, in the order defined by its proximity list. If all the pools are full, the producer invokes the \emph{produceForce} operation on the closest SCPool, which always succeeds (expanding the pool if needed). 
	\item {\bf Consumer's policy.} A consumer consumes tasks from its own SCPool. If its SCPool is empty, the consumer tries to steal tasks from other pols in the order defined by its proximity list. 
\end{itemize}



% The inter-pool communication policy is a subject to engineering optimizations and its optimal behavior should probably
% depend on the workload. For the purpose of our evaluation we propose the following approach. 
	% Producer policy. Each producer is provided with the list of all available consumers sorted according to the locality considerations of the given architectures. For example, in case of 
% In order to insert a task a producer first invokes a produce() operation on the closest consumer. If this operation fails, 
% then the closest consumer's pool is full (which could be evidence of an over-load of the given consumer thread) and the producer should 
% try to insert a task to another consumer. If neither consumer ... a producer finally invokes produceForce(), which expands the pool if necessary and always succeeds to insert the task. 
	% Consumer policy. A consumer works in a loop of consuming its own tasks. If the own pool of a consumer is empty, the consumer iterates over all other consumers and tries to steal tasks from there. 




\section{Algorithm Description}
\label{sec:algo}
In the current section we present the SALSA SCPool. We first show the data structures of SALSA in Section~\ref{alg-structure}, and then present the basic algorithm without stealing support in Section~\ref{alg-overview}. The stealing procedure is described in Section~\ref{alg-stealing}, finally, the role of chunk pools is presented in Section~\ref{alg-pools}.

\subsection{SALSA Structure\label{alg-structure}}


\begin{figure}[htb]
	\centering
	\includegraphics[height=0.3\textwidth]{figures/salsa-struct}
	\caption{
	    \footnotesize{Overview of the SALSA per-consumer pool. Each producer has a list of
nodes that points to chunks. Empty node are lazily removed. The last list is used for chunks that
the consumer stole from other SALSA pools. The index in each node indicates the location of the
next available task in the chunk and is used when stealing}}
	\label{fig:salsa-struct}
\end{figure}

In this section we describe the structure of the SALSA implementation of a single-consumer pool.
The structure is described in Algorithm~\ref{alg:non-fifo-ds} and Figure~\ref{fig:salsa-struct}

The structure is composed of number of producers + 1 lists of nodes. The first lists corresponds to
the producers, and the last list is used for stolen chunks. Each node has a pointer to a
chunk and an index (idx), this indicates the index of the last task already taken or the index of
a task which is about to be taken by the consumer, this field is used by a consumer to know what is
the next available task in the chunk and it is also used when stealing chunks (see
Section~\ref{alg-stealing}).
Chunks are arrays of pointers to tasks. When there is no task in a cell it contains $\bot$, after
the task was taken by a consumer, the cell will contain the special value \emph{TAKEN}. Each chunk
has an owner which is the ID of the consumer that is currently working with that chunk. 

The lists are single-writer multiple reader linked-lists, similar to the one shown
in~\cite{Michael:2004:HPS:987524.987595}. Producer $i$ is owner of list $i$ and the consumer is the
owner last list, the only one to modify the list is the owner. Nodes are removed lazily: when a list
owner sees a node that does not point to a chunk, this node is removed and reclaimed. When a
producer wishes to add a new chunk to the list, he traverses the list, removing nodes without chunks
while doing so, then he appends to the list a new node that points to the new chunk. Hazard
pointers~\cite{Michael:2004:HPS:987524.987595} are used to manage the reclamation of nodes and
chunks and solve the ABA problem. 

Each Consumer has a chunk pool, which is implemented using a
lock-free queue~\cite{Michael:1996:SFP:248052.248106}. These pools are used to enable reuse of
chunks, and also so producers can realize when a consumer is full and move to another consumer (see
Section~\ref{alg-pools}).

\subsection{Basic algorithm\label{alg-overview}}
\subsubsection {SALSA producer overview}
\begin{algo}[!ht]
\caption{SALSA implementation of SCPool: Producer Functions.}
\label{alg:producer-non-fifo}
\scriptsize
\begin{minipage}[t]{0.48\textwidth}
\begin{distribalgo}[1]
\setcounter{ALC@line}{\value{alg:non-fifo:lines}}

\INDENT {{\bf Producer local variables}:}
	\STATE int producerId
	\STATE Chunk chunk; initially $\bot$ \comment {the chunk to insert to}
	\STATE int prodIdx; initially $0$ \comment {the prefix of inserted tasks}
\ENDINDENT

\medskip

\INDENT {{\bf Function produce}(Task t, SALSA scPool):}
	\STATE {\bf return insert}(t, scPool, false)
\ENDINDENT

\medskip

\INDENT {{\bf Function insert}(Task t, SCPool scPool, bool force):}
	\INDENT {{\bf if} (chunk $= \bot$) {\bf then}}
		\STATE {\bf if} ({\bf getChunk}(scPool, force) $=$ {\bf FULL}) {\bf then return FULL}
	\ENDINDENT
	\STATE chunk.tasks[prodIdx] $\leftarrow$ t; prodIdx++
	\INDENT {{\bf if}(prodIdx $=$ CHUNK\_SIZE) {\bf then}}
	  \STATE chunk $\leftarrow \bot$ \comment {the chunk is full}
	\ENDINDENT
	\STATE {\bf return SUCCESS}
\ENDINDENT

\setcounter{alg:non-fifo:lines}{\value{ALC@line}} % store the line number
\end{distribalgo}
\end{minipage}%
%
\hfill
%
\begin{minipage}[t]{0.48\textwidth}
%
\begin{distribalgo}[1]
\setcounter{ALC@line}{\value{alg:non-fifo:lines}}

\INDENT {{\bf Function produceForce}(Task t, SALSA scPool):}
	\STATE {\bf return insert}(t, scPool, true)
\ENDINDENT

\medskip

\INDENT {{\bf Function getChunk}(SALSA scPool, bool force)}
	\STATE newChunk $\leftarrow$ a chunk from scPool.chunkPool
	\INDENT {{\bf if} (chunk $= \bot$) \comment {no available chunks in this pool}}
		 \STATE {\bf if}(force $=$ false) {\bf then return FULL} 
		 \STATE newChunk $\leftarrow$ allocate a new chunk
	\ENDINDENT
	\STATE newChunk.owner $\leftarrow$ scPool.consumerId
	\STATE node $\leftarrow$ new node with idx $=-1$ and c $=$ newChunk
	\STATE scPool.chunkLists[producerId].{\bf append}(node)
	\STATE chunk $\leftarrow$ newChunk; prodIdx $\leftarrow 0$ 
	\STATE {\bf return SUCCESS}
\ENDINDENT

\setcounter{alg:non-fifo:lines}{\value{ALC@line}}
\end{distribalgo}
\end{minipage}
\end{algo}

The description of SALSA producer functions is presented in Algorithm~\ref{alg:producer-non-fifo}. 
The insertion of a new task consists of two stages: 
1) finding a chunk for task insertion, and 2) adding a task to the chunk. 

\paragraph {Finding a chunk.}
The chunk for task insertions is kept at the local producer variable \emph{chunk} (line~\ref{alg:line:chunk} in Algorithm~\ref{alg:producer-non-fifo}). 
Once a producer starts working with a chunk $c$, it continues putting the tasks to $c$ until it is full. 
Note that the producer is oblivious to chunk stealing and is not aware of chunks moving from one pool to another. 
If the \emph{chunk}'s value is $\bot$, then the producer should find a new chunk (function \emph{getChunk}). 
In this case, the it tries to retrieve a chunk from the chunk pool and append it to the appropriate chunk list. If the chunk pool is empty then the producer either allocates a new chunk by itself (\emph{produceForce()} function), or returns $\bot$ (\emph{produce()} function) (lines~\ref{alg:line:no-chunk-start}--\ref{alg:line:no-chunk-end}). 

\paragraph {Inserting a task to the chunk.}
As previously described in Section~\ref{alg-structure}, different producers insert tasks to different chunks, which removes the need for synchronization among producers. 
The producer local variable \emph{prodIdx} indicates the next free slot in the chunk.
All that the insertion function has to do is to put a task in that slot and to increment \emph{prodIdx} (line~\ref{alg:line:chunk-insert}).
Once the index reaches the maximal value, the \emph{chunk} variable is set to $\bot$, indicating that the next insertion operation should start a new chunk. 

\subsubsection {SALSA consumer overview}
\begin{algo}[!ht]
\caption{SALSA implementation of SCPool: Consumer Functions.} 
\label{alg:non-fifo}
\scriptsize
\begin{minipage}[t]{0.48\textwidth}
\begin{distribalgo}[1]
\setcounter{ALC@line}{\value{alg:non-fifo:lines}}
\smallskip


\INDENT {{\bf Function consume}():}
  \INDENT {{\bf if} (currentNode $\neq \bot$) {\bf then} \comment {common case}}
		\STATE t $\leftarrow$ {\bf takeTask}(currentNode)
		\STATE {\bf if} (t $\neq \bot$) {\bf then return} t
	\ENDINDENT
	\INDENT {{\bf foreach} Node $n$ in ChunkLists {\bf do:} \comment {fair traversal of chunkLists}}
	  \INDENT {{\bf if} (n.c $\neq \bot \wedge \textrm{n.c.owner} = \textrm{consumerId}$) {\bf then}}
  			\STATE t $\leftarrow$ {\bf takeTask}(n)
				\STATE {\bf if} (t $\neq \bot$) {\bf then} currentNode $\leftarrow$ n; {\bf return} t
			\ENDINDENT
	\ENDINDENT
% 	\comment {Iterate over other chunk lists}
% 	\INDENT {{\bf foreach} cl {\bf in} chunkLists {\bf do} \comment {fair traversal}} 
%   	\INDENT {{\bf foreach} node {\bf in} cl {\bf do}}
%   	  \INDENT {{\bf if}(node.c $\neq \bot \wedge \textrm{node.c.owner} = \textrm{consumerId}$) {\bf then}}
%   			\STATE t $\leftarrow$ {\bf takeTask}(node)
% 				\STATE {\bf if} (t $\neq \bot$) {\bf then} currentNode $\leftarrow$ node; {\bf return} t
% 			\ENDINDENT
%   	\ENDINDENT
% 	\ENDINDENT
	\STATE currentNode $\leftarrow \bot$; {\bf return} $\bot$
\ENDINDENT

\medskip

\INDENT {{\bf Function takeTask}(Node n):}
  \STATE chunk $\leftarrow$ n.c
  \STATE {{\bf if} (chunk $= \bot$) {\bf then return $\bot$} \comment{this chunk has been stolen}}
 
  \STATE task $\leftarrow$ chunk.tasks[n.idx + 1]
  \STATE {\bf if} (task $= \bot$) {\bf then return} $\bot$ \comment{no inserted tasks}
 	
 	\smallskip 
  \STATE n.idx++ \comment {tell the world you're going to take a task from idx} \label{alg:lines:ind-inc}
  \INDENT {{\bf if} (chunk.owner $=$ consumerId) {\bf then} \comment {common case}}
 		\STATE chunk.tasks[n.idx] $\leftarrow$ TAKEN \label{alg:lines:fast-path}
  	\STATE {\bf checkEmpty}(n)
  	\STATE {\bf return} task 
  \ENDINDENT
  
  \smallskip
  \comment {the chunk has been stolen, CAS the last task and go away} 
 	\STATE success $\leftarrow$ (task $\neq$ TAKEN $\wedge$ \\ \label{alg:lines:stolen-chunk-begin}
 		\hspace{0.5cm} CAS(chunk.tasks[n.idx], task, TAKEN)) \label{alg:line:cas-consumer}
 	\STATE {\bf if}(success) {\bf then checkEmpty}(n) 
	\STATE currentNode $\leftarrow \bot$
 	\STATE {\bf return} (success) ? task : $\bot$ \label{alg:lines:stolen-chunk-end}
\ENDINDENT



\medskip

\INDENT {{\bf Function checkEmpty}(Node n):}
	\INDENT {{\bf if}(n.idx + 1 $=$ CHUNK\_SIZE) {\bf then} \comment {finished the chunk}}
  	\STATE n.c $\leftarrow \bot$; return chunk to chunkPool
  	\STATE currentNode $\leftarrow \bot$
  \ENDINDENT
\ENDINDENT


\setcounter{alg:non-fifo:lines}{\value{ALC@line}} % store the line number
\end{distribalgo}
\end{minipage}%
%
\hfill
%
\begin{minipage}[t]{0.48\textwidth}
%
\begin{distribalgo}[1]
\setcounter{ALC@line}{\value{alg:non-fifo:lines}}
\smallskip

\INDENT {{\bf Function steal}(SCPool from):}
	\STATE prevNode $\leftarrow$ a node holding tasks from some list \comment {different policies possible} \label{alg:line:take-steal-chunk}
	\STATE c $\leftarrow$ prevNode.c; {\bf if} (c = $\bot$) {\bf then return} $\bot$

	\STATE chunkLists[steal].{\bf append}(prevNode) \comment {make it restealable} \label{alg:line:resteal-begin}
	\INDENT {{\bf if} ({\bf CAS}(c.owner, from.consumerId, consumerId) $=$ false)} \label{alg:line:chown}
 		\STATE chunkLists[steal].{\bf remove}(prevNode)
 		\STATE {\bf return} $\bot$ \comment {failed to steal}
	\ENDINDENT

	\smallskip
	\STATE newNode $\leftarrow$ copy of prevNode \label{alg:line:copy-prev-node}
	\INDENT {{\bf if} (newNode.idx+1 $=$ CHUNK\_SIZE) \comment {Chunk is empty}} \label{alg:line:steal-node-empty}
	  \STATE chunkLists[steal].{\bf remove}(prevNode)
	  \STATE {\bf return} $\bot$
	\ENDINDENT
	\STATE replace prevNode with newNode in chunkLists[steal]
	\STATE prevNode.c $\leftarrow \bot$ \comment {remove chunks from consumer's list} \label{alg:line:remove-chunk} \label{alg:line:resteal-end}
	
	\smallskip
	\comment {done stealing the chunk, take one task from it}
	\STATE idx $\leftarrow$ newNode.idx
	\STATE task $\leftarrow$ c.tasks[idx+1] 
	\STATE {\bf if} (task $= \bot$) {\bf then return} $\bot$ \comment {still no task at idx+1} \label{alg:line:steal-chunk-not-full}
	\INDENT {{\bf if} (task $=$ TAKEN $\vee$ \\
		\hspace{0.5cm} !{\bf CAS}(c.tasks[idx+1], task, TAKEN)) {\bf then}} \label{alg:line:cas-steal} 
		\STATE task $\leftarrow \bot$
	\ENDINDENT
	\STATE currentNode $\leftarrow$ newNode
	\STATE {\bf if} (task $\neq \bot$) {\bf then checkEmpty}(newNode)
	\STATE newNode.idx $\leftarrow$ newNode.idx+1
	\STATE {\bf return} task
\ENDINDENT

\setcounter{alg:non-fifo:lines}{\value{ALC@line}}
\end{distribalgo}
\end{minipage}
\end{algo}


We now describe the way the consumer retrieves a task from the chunk in the fast path with no stealing. 
Unlike producers that operate solely on the chunks, a consumer should take into consideration the possibility of stealing. 
Therefore, it should notify other processes about the task it is about to take. 

Each node in the chunk list keeps an index of the taken prefix of its chunk (\emph{idx} variable, which is initiated to $-1$). 
A consumer that wants to take a task $T$ first increments the index, checks the ownership and then changes the chunk entry from $T$ to a special value \emph{TAKEN}. By doing this, a consumer guarantees that \emph{idx} always points to the last taken task or to a task that is about to be taken. Hence, a process that is stealing a chunk from the node with $\textit{idx} = i$ can assume that the tasks in the range $[0 \ldots i)$ have already been taken.




\subsection{Stealing\label{alg-stealing}}
\snegspace
The stealing algorithm is given in the function {\bf steal()} in Algorithm~\ref{alg:non-fifo}. 
We refer to the stealing consumer as $c_s$, the victim process whose chunk is being stolen as $c_v$, and the stolen chunk as $ch$.

The idea is to turn $c_s$ to the exclusive owner of $ch$, such that $c_s$ will be able to take tasks from the chunk without synchronization. 
In order to do that, $c_s$ changes the ownership of $ch$ from $c_v$ to $c_s$ using CAS (line~\ref{alg:line:chown}) and removes the chunk from $c_v$'s list (line~\ref{alg:line:remove-chunk}). 
Once $c_v$ notices the change in the ownership it can take at most one more task from $ch$ (lines~\ref{alg:lines:stolen-chunk-begin}--\ref{alg:lines:stolen-chunk-end}). 

When the {\bf steal()} operation of $c_s$ occurs simultaneously with the {\bf takeTask()} operation of $c_v$, both $c_s$ and $c_v$ might try to retrieve the same task. We now explain why this might happen. Recall that $c_v$ notifies potential stealers of the task it is about to take by incrementing the \emph{idx} value in $ch$'s node (line~\ref{alg:lines:ind-inc}). This value is copied by $c_s$ in line~\ref{alg:line:copy-prev-node} when creating a copy of $ch$'s node for its steal list.

Consider, for example, a scenario in which the $idx$ is incremented by $c_v$ from $10$ to $11$. 
If $c_v$ checks $ch$'s ownership before it is changed by $c_s$, then $c_v$ takes the task at index $11$ \emph{without synchronization} (line~\ref{alg:lines:fast-path}). Therefore, $c_s$ cannot be allowed to take the task pointed by \emph{idx}. Hence, $c_v$ has to take the task at index $11$ even if it does observe the ownership change. 
After stealing the chunk, $c_s$ will eventually try to take the task pointed by $idx+1$. However, if $c_s$ copies the node before $idx$ is incremented by $c_v$, $c_s$ might think that the value of $idx+1$ is $11$. In this case, both $c_s$ and $c_v$ will try to retrieve the task at index $11$. To ensure that the task is not retrieved twice, both invoke CAS in order to retrieve this task (line~\ref{alg:line:cas-steal} for $c_s$, line~\ref{alg:line:cas-consumer} for $c_v$). 

The above algorithm works correctly as long as the stealing consumer can observe the node with the updated index value. 
This might not be the case if the same chunk is concurrently stolen by another consumer rendering the \emph{idx} of the original node obsolete. 
In order to prevent this situation, stealing a chunk from the pool of consumer $c_v$ is allowed only if $c_v$ is the owner of this chunk (line~\ref{alg:line:chown}). This approach is prone to the ABA problem: consider a scenario where consumer $c_a$ is trying to steal from $c_b$, but before the execution of the CAS in line~\ref{alg:line:chown}, the chunk is stolen by $c_c$ and then stolen back by $c_b$. In this case, $c_a$'s CAS succeeds but $c_a$ has an old value of $idx$. To prevent this ABA problem, the owner field contains a ``tag'', which is incremented on every CAS operation. For brevity, tags are omitted from the pseudo-code.

A na\"{i}ve way for $c_s$ to steal the chunk from $c_v$ would be first to change the ownership and then to move the chunk to the steal list. However, this approach may cause the chunk to ``disappear'' if $c_s$ is stalled, because the chunk becomes inaccessible via the lists of $c_s$ and yet $c_s$ is its owner. 
%In this case, as we explained above, the chunk cannot be stolen, which prevents other consumers from taking tasks from this chunk. 
Therefore, SALSA first adds the original node to the steal list of $c_s$, then change the ownership, and only then replaces the original node with a new one (lines~\ref{alg:line:resteal-begin}--\ref{alg:line:resteal-end}).
\subsection{Chunk pools\label{alg-pools}}
As described in Section~\ref{alg-structure}, each consumer has a pool of chunks of a limited size.
When a producer needs a new chunk to add to its list in consumer $c_i$, it tries to get a chunk from $c_i$'s chunk pool. If the pool doesn't contain any chunks, the the operation fails (if {\it produce}() was invoked) or it creates a new chunk (if {\it produceForce}() was invoked). As described in Section~\ref{sec:system}, our policy is first to invoke {\it produce}() on all the pools by order of distance from the producer, and only after the producer fails to insert the task to all pools it calls {\it produceForce}() on the closest pool. This policy reduces the number of steal operation by making producers balance the load, rather than the consumers. In addition, in SALSA a chunk is returned to the pool of the consumer which took the last task in that chunk. This balances the pool sizes, by making pools of overloaded consumers smaller and thus cause producers to move to other consumers.

\section{Correctness}
\label{sec:correctness}
\begin{figure}[htb]
	\centering
	\includegraphics[width=0.7\textwidth]{figures/linearizability-example}
	\caption{\footnotesize{An example where a single traversal may violate linearizability: consumer $a$ is trying to get a task. It fails to take a task from its own pool, and starts looking for chunks to steal in other pools. At this time there is a single non-empty chunk in the system, which is in $b$'s pool; $a$ checks $c$'s pool and finds it empty. At this point, a producer adds a task to $c$'s pool and then $b$ takes the last task from its pool before $a$ checks it. Thus, $a$ finds $b$'s pool empty, and returns $\bot$. There is no way to linearize this execution, because throughout the execution of $a$'s operation, the system contains at least one task.}}
	\label{fig:linearizability-example}
\end{figure}

\paragraph{Linearizability.}
SALSA pools only return tasks that were inserted to the pool and do not return the same task twice (see Appendix~\ref{appendix:salsa-correctness}). However, they are not linearizable by themselves since a {\bf consume()} or {\bf steal()} operation may return $\bot$, even when the pool is not empty. For our system to be linearizable, we must ensure that it will return $\bot$ only if it is empty (i.e., contain no tasks) at some point during the consume operation. We describe a policy for doing so in a lock-free manner. 

Let us examine why a na\"ive approach, of simply traversing all task pools and returning $\bot$ if no task is found, violates correctness. First, a consumer might ``miss'' one task added during its traversal, and another removed during the same traversal, as illustrated in Figure 3. In this case, a single traversal would have returned $\bot$ although the pool was not empty at any point during the consume operation. Second, a consumer may miss a task that is moved from one pool to another due to stealing. In order to identify these two cases, we add to each pool a special \emph{emptyIndicator}, a bit array with a bit per-consumer, which is cleared every time the pool \emph{may} become empty. In SALSA, this occurs when the last task in a chunk is taken or when a chunk is stolen. 
In addition, we implement a new function, {\bf checkEmpty()}, which is called by the framework whenever a consumer fails to retrieve tasks from its pool and all other pools. This function return true only if there is a time during its execution when there are no tasks in the system. If {\bf checkEmpty()} returns false, the consumer simply restarts its operation. 

Denote by $c$ the number of consumers in the system. The {\bf checkEmpty()} function works as follows: the consumer traverses all SCPools, to make sure that no tasks are present. After checking a pool, the consumer sets its bit in \emph{emptyIndicator} using CAS. The consumer repeats this traversal $c$ times, where in all traversals except the first, it checks that its bit in \emph{emptyIndicator} is set, i.e., that no chunks were emptied or removed during the traversal. The $c$ traversals are needed in order to account for the case that other consumers have already stolen or removed tasks, but did not yet update \emph{emptyIndicator}, and thus their operations were not detected by the consumer. Since up to $c-1$ pending operations by other consumers may empty pools before any \emph{emptyIndicator} changes, it is guaranteed that among $c$ traversals in which no chunks were seen and the \emph{emptyIndicator} did not change, there is one during which the system indeed contains no tasks, and therefore it is safe to return $\bot$. This method is similar to the one used in Concurrent Bags~\cite{Sundell:2011:LAC:1989493.1989550}.
\paragraph{Lock-freedom.}
The operations of every individual SALSA SCPool are trivially wait-free, since they always return. However, a consume operation is restarted whenever {\bf checkEmpty()} returns false, and therefore the algorithm does not guarantee that a consumer will finish every operation. Nevertheless, the system is lock-free, i.e., there always exists some consumer that makes progress. Since 1) whenever all pools are empty for sufficiently long {\bf checkEmpty()} returns true, and 2) a consumer either returns a task or keeps attempting to steal tasks, lock-freedom immediately follows from the following claim:

\begin{claim}
\label{claim:lock-free}
If a consumer returns $\bot$ in $c$ steal attempts from non-empty pools (i.e., pools that contain a task when the steal operation starts), then at least one consumer returns a task during that time. 
\end{claim}
The proof for this claim appears in Appendix \ref{appendix:lock-freedom}.

\section{Implementation and Evaluation}
\label{sec:evaluation}
In this section we evaluate the performance of our work-stealing framework built on SALSA pools. 
We describe the experiment setup in Section~\ref{sec:exp-setup}, we show the overall system performance in Section~\ref{sec:eval-performance} and study the influence of various SALSA techniques in Section~\ref{sec:eval-techniques}.

\subsection {Experiment Setup}
\label{sec:exp-setup}
The implementation of the work-stealing framework used in our evaluation does not include the linearizability mechanism described in~\ref{sec:correctness}. We believe that this mechanism has negligible effect on performance; moreover, in our experiment they would not have been invoked because the pool is never empty. We compare the following task pool implementations:
\begin {itemize}
\item
{\bf SALSA} -- our work-stealing framework with SCPools implemented by SALSA.
\item
{\bf SALSA+CAS} -- our work-stealing framework with SCPools implemented by a simplistic SALSA variation, in which every {\bf consume()} and {\bf steal()} operation tries to take a single task using CAS. In essence, SALSA+CAS removes the effects of SALSA's low-synchronization fast-path and per-chunk stealing. 
Note that disabling per-chunk stealing in SALSA annuls the idea of chunk ownership, hence, disables its low-synchronization fast-path as well. 
\item
{\bf ConcBag} -- an algorithm similar to the lock-free Concurrent Bags algorithm~\cite{Sundell:2011:LAC:1989493.1989550}. 
It is worth noting that the original algorithm was optimized for the scenario where the same process is both a producer and a consumer (in essence producing tasks to itself), which we do not consider in this paper; in our system no thread acts as both a producer and a consumer, therefore every consume operation steals a task from some producer.
We did not have access to the original code, and therefore reimplemented the algorithm in our framework. Our implementation is faithful to the algorithm in the paper, except in using a simpler and faster underlined linked list algorithm. All engineering decisions were made to maximize performance. 
\item
{\bf WS-MSQ} -- our work-stealing framework with SCPools implemented by Michael-Scott non-blocking queue~\cite{Michael:1996:SFP:248052.248106}. Both {\bf consume()} and {\bf steal()} operations invoke the {\bf dequeue()} function. 
\item
{\bf WS-LIFO} -- our work-stealing framework with SCPool implemented by Michael's LIFO stack~\cite{Michael:2004:HPS:987524.987595}. 
\end {itemize} 

We did not experiment with additional FIFO and LIFO queue implementations, because, as shown in~\cite{Sundell:2011:LAC:1989493.1989550}, their performance is of the same order of magnitude as the Michael-Scott queue. 
Similarly, we did not evaluate {CAF\'E}\cite{Basin:2011:CST:2075029.2075087} pools because their performance is similar to that of WS-MSQ~\cite{Basin:Thesis:2011}, or ED-Pools~\cite{Afek:2010:SPP:1885276.1885295}, which have been shown to scale poorly in multi-processor architectures~\cite{Basin:Thesis:2011,Sundell:2011:LAC:1989493.1989550}. 

All the pools are implemented in C++ and compiled with \texttt{-O2} optimization level. 
In order to minimize scalability issues related to allocations, we use \texttt{jemalloc} allocator~\cite{citeulike:4951109}, which has been shown to be highly scalable in multi-threaded environments\footnote{\url{http://www.facebook.com/notes/facebook-engineering/scalable-memory-allocation-using-jemalloc/480222803919}}.
Chunks of SALSA and SALSA+CAS contain $1000$ tasks, and chunks of ConcBag contain $128$ tasks, which were the respective optimal values for each algorithm (see Appendix \ref{appendix:chunk-size}). 

We use a synthetic benchmark where 1) each producer works in a loop of inserting dummy items; 2) each consumer works in a loop of retrieving dummy items. Each data point shown is an average of $5$ runs, each with a duration of $20$ seconds. 
The tests are run on a dedicated shared memory NUMA server with $8$ Quad Core AMD $2.3$GHz processors and $16$GB of memory attached to each processor. 

\subsection{System Throughput}
\label{sec:eval-performance}

\begin{figure}[htb]
	\centering
  \subfigure [\scriptsize{System throughput -- N producers, N consumers.}] {
    \includegraphics[width=0.45\textwidth]{figures/n-n-throughput}
    \label{fig:n-n-throughput}
  }
  \subfigure [\scriptsize{System throughput -- variable producers-consumers ratio.}] {
    \includegraphics[width=0.45\textwidth]{figures/prod-cons-fig}
    \label{fig:prod-cons-fig}
  }
	\caption{\footnotesize{System throughput for various ratios of producers and consumers. SALSA scales linearly with the number of threads -- in the $16/16$ workload, it is $\times20$ faster than WS-MSQ and WS-LIFO, and more than $\times5$ faster than Concurrent Bags. In tests with equal numbers of producers and consumers, the differences among work-stealing alternatives are mainly explained by the consume operation efficiency, since stealing rate is low and hardly influences performance. In Concurrent Bags every {\bf consume()} operation implies stealing, which causes its sub-linear scalability.
}}
	\label{fig:throughput}
\end{figure}

Figure~\ref{fig:throughput} demonstrates system throughput of the compared algorithms for workloads with various numbers of producers and consumers. 
Figure~\ref{fig:n-n-throughput} shows throughput dynamics when the number of producers is equal to the number of consumers. SALSA \emph{scales linearly} as the number of threads grows to $32$ (the number of physical cores in the system), and it clearly outperforms all other competitors. In the $16/16$ workload, SALSA is $\times20$ faster than WS-MSQ and WS-LIFO, and more than $\times5$ faster than Concurrent Bags. 

We note that the performance trend of ConcBags in our measurements differs from the results presented by Sundell et al.~\cite{Sundell:2011:LAC:1989493.1989550}. 
While in the original paper, their throughput \emph{drops} by a factor of $3$ when the number of threads increases from $4$ to $24$, in our tests, the performance of ConcBags \emph{increases} with the number of threads. The reasons for the better scalability of our implementation compared of the original one can be related to the use of different memory allocators, hardware architectures, and engineering optimizations. %In any case, both implementations provide the performance of tens of thousands of task retrievals in msec for multiple producers and consumers. 

The stealing rate in workloads with equal numbers of producers and consumers remains low; graphs showing these results are omitted due to space limitations. The performance differences are therefore due to the efficiency of the {\bf consume()} operation. 
For example, SALSA is $\times1.7$ faster than SALSA+CAS thanks to the fast-path consumption technique, which does not use strong atomic operations. The impact of CAS operations is fairly modest, however, since they seldom compete.
In contrast, in ConcBags, which is not based on per-consumer pools, every {\bf consume()} operation implies stealing, which causes contention among consumers, leading to sub-linear scalability.

Figure~\ref{fig:prod-cons-fig} shows system throughput of the algorithms for various ratios of producers and consumers. 
Our algorithm achieves maximal throughput for the equal number of producers and consumers because neither of them is a system bottleneck.
SALSA outperforms other alternatives for all possible producer-consumer ratios. 

\begin{figure}[htb]
	\centering
  \subfigure [\scriptsize{System throughput -- 1 Producer, N consumers.}] {
    \includegraphics[width=0.45\textwidth]{figures/1-n-throughput}
    \label{fig:1-n-throughput}
  }
  \subfigure [\scriptsize{CAS operations per task retrieval -- 1 Producer, N consumers.}] {
    \includegraphics[width=0.45\textwidth]{figures/1-n-cas}
    \label{fig:1-n-cas}
  }
	\caption{\footnotesize{System behavior in workloads with a single producer and multiple consumers. 
	Both SALSA and SALSA+CAS efficiency balance the load in this scenario. The throughput of other algorithms drops by a factor of $10$ due to increased contention among consumers trying to steal tasks from the same pool.}}
	\label{fig:1-n-perf}
\end{figure}

We next evaluate the behavior of the pools in scenarios with a single producer and multiple consumers. 
Figure~\ref{fig:1-n-throughput} shows that for both SALSA and SALSA+CAS the performance does not drop as more consumers are added. In contrast, the throughput of other algorithms drops by the factor of $10$ when the number of consumers rises from $1$ to $32$. 
This degradation is caused by the increased contention among consumers that try to steal tasks from the same pool, as evident from Figure~\ref{fig:1-n-cas}, which shows the average number of CAS operations per task transfer (by both producers and consumers). As we shall see in the next section, SALSA+CAS's low contention is achieved thanks to the producer-based balancing described in Section~\ref{alg-pools}, and SALSA achieves significantly better throughput thanks to chunk-based stealing.

\subsection{Evaluation of SALSA Techniques}
\label{sec:eval-techniques}
\begin{figure}[htb]
	\centering
  \subfigure [\scriptsize{System throughput -- 1 Producer, N consumers.}] {
    \includegraphics[width=0.45\textwidth]{figures/1-n-salsa}
    \label{fig:1-n-salsa-perf}
  }
  \subfigure [\scriptsize{System throughput as a function of stalled threads ($16/16$ workload). Black dashed lines show the theoretical throughput degradation proportional to the number of stalled threads.}] {
    \includegraphics[width=0.45\textwidth]{figures/stalled-threads}
    \label{fig:stalled-threads}
  }
	\caption{\footnotesize{The effects of chunk-based stealing and producer-based balancing on system throughput in imbalanced scenarios. Producer-based balancing contributes to the robustness of the framework by reducing stealing. With no balancing, chunk-based stealing becomes important. }}
	\label{fig:1-n-salsa}
\end{figure}
In this section we study the influence of two of the techniques used in SALSA: 1) chunk-based-stealing with a low-synchronization fast path (Section~\ref{alg-stealing}), and 2) producer-based balancing (Section~\ref{alg-pools}). 
To this end, we compare SALSA and SALSA+CAS both with and without producer-based balancing (in the latter a producer always inserts tasks to the same consumer's pool).

Figure~\ref{fig:1-n-salsa-perf} depicts the behavior of the four alternatives in single producer / multiple consumers workloads. 
We see that producer-based balancing is instrumental in redistributing the load: neither SALSA nor SALSA+CAS suffers any degradation as the load increases. 
When producer-based balancing is disabled, stealing becomes prevalent, and hence the stealing granularity becomes more important: 
SALSA's chunk based stealing clearly outperforms the na\"{i}ve task-based approach of SALSA+CAS. 

Figure~\ref{fig:stalled-threads} shows the case where an equal number of producer and consumer threads are stalled in $16/16$ workload (e.g., if $4$ threads are stalled then there are $2$ paused producers and $2$ paused consumers). This simulates the scenario of an overloaded machine in which some threads can be starved for long periods of time, or a scenario where some threads are busy running excessively long tasks. 
The stalled threads are chosen so that the default producers of the frozen consumers are not stalled, which leads to imbalance in the number of tasks among the SALSA pools. 
Reducing the number of participating threads inherently degrades performance; the black dashed lines indicate the theoretical performance degradation in proportion to the number of stalled threads.
The graphs demonstrate that producer-based balancing contributes to the robustness of the framework, and allows both SALSA variants to achieve performance close to that of the theoretical bound.

When producer-based balancing is disabled, a high stealing rate is inevitable, which causes a severe throughput degradation.

%
%\begin{wrapfigure}{r}{0.47\textwidth}
%  \vspace{-20pt}
%  \begin{center}
%    \includegraphics[width=0.45\textwidth]{figures/stalled-threads}
%  \end{center}
%  \vspace{-20pt}
%  \caption{\footnotesize{System throughput in a system with $16$ producers and $16$ consumers as a function of the number of stalled threads.}}
%  \vspace{-10pt}
%  \label{fig:stalled-threads}
%\end{wrapfigure}
%
%%
%%
%%\begin{figure}[htb]
%%	\centering
%%	\includegraphics[width=0.45\textwidth]{figures/stalled-threads}
%%  \caption{\footnotesize{System throughput for different number of stalled threads (N/N workload). }}
%%	\label{fig:stalled-threads}
%%\end{figure}
%
%producer migration -- robustness against unpredictable stalls
%
%\begin{figure}[htb]
%	\centering
%	\includegraphics[width=0.45\textwidth]{figures/chunk-size}
%  \caption{\footnotesize{System throughput as a function of chunk size. }}
%	\label{fig:chunk-size}
%\end{figure}

\section{Conclusions}
\label{sec:conclusions}
We presented a highly-scalable task pool framework, built upon
our novel SALSA single-consumer pools and work stealing.
Our work has employed a number of novel techniques for improving
performance: 1) lightweight and synchronization-free produce and consume operations in the common case; 
2) NUMA-aware memory management, which keeps most data accesses inside NUMA nodes;
3) a chunk-based stealing approach that decreases the stealing cost and suits NUMA migration schemes; and 4) elegant producer-based
balancing for decreasing the likelihood of stealing.

We have shown that our solution scales linearly with the number
of threads. It outperforms other work-stealing techniques by a
factor of $20$, and state-of-the art non-FIFO pools by a factor of $3.5$.
We have further shown that it is highly robust to imbalances and
unexpected thread stalls.

We believe that our general approach of partitioning data structures
among threads, along with chunk-based migration and an efficient
synchronization-free fast-path, can be of benefit in building
additional scalable high-performance services in the future.
\bibliographystyle{abbrv}
\bibliography{refs}

%\begin{appendix}
%
%\section{SALSA correctness}
%\subsection{Lock-freedom}
%\label{appendix:lock-freedom}
%As we show on Section~\ref{sec:correctness}, in order to show that our system is lock-free, it is enough to prove the following claim:

\newtheorem*{claim:lock-free}{Claim \ref{claim:lock-free}}
\begin{claim:lock-free}
If a consumer returns $\bot$ in $c$ steal attempts from non-empty pools (i.e., pools that contain a task when the steal operation starts), where $c$ is the number of consumers in the system, then at least one consumer in the system returns a task during that time. 
\end{claim:lock-free}
\begin{proof}
We argue that the above claim is true SALSA with the modifications describe above. Consider a consumer that invokes {\bf steal()} on a non-empty pool. This operation may fail to return a task in four cases:
\begin{enumerate}
 \item The {\bf if} statement in line~\ref{alg:line:steal-node-empty} is true. In this case all the tasks were already taken from this chunk. However, in line~\ref{alg:line:take-steal-chunk}, the steal operation took a chunk containing tasks, and therefore tasks were taken from this chunk (by other consumers) during the steal operation.
 \item The {\bf if} statement in line~\ref{alg:line:cas-steal} is true. This may occur because task$=$TAKEN or because the CAS fails. In both cases, it means that the task pointed to by $idx+1$ was taken by another consumer during the operation.
 \item The {\bf if} statement in line~\ref{alg:line:steal-chunk-not-full} is true. This means that the next available task pointed by $idx+1$ is a free slot in the chunk. However when the task was chosen in line~\ref{alg:line:take-steal-chunk}, the chunk contained a task, i.e., $idx+1$ pointed to a non-empty slot. This means that $idx$ was increased by the owner of the pool and so by the time its operation completes, it either returns the task or fails in the CAS because the task has already been returned.
 \item The {\bf if} statement in line~\ref{alg:line:chown} is true, In this case, the steal operation fails because another consumer steals this chunk. Since the consumer that stole this chunk succeed in his CAS in line~\ref{alg:line:chown}, the current case can not apply for it and therefore its operation either returns a task, or one of the above cases applies to it and therefore there is another operation that upon its completion a task will be returned. 
 \item The {\bf if} statement in line~\ref{alg:line:check-steal-chunk-null} is true, In this case, no stealable chunks are found (i.e., chunks that contain a task and its owner is the owner of the current pool). Since we assumed the the pool contains a task, this mean that either a concurrent consume operation returned a task or a another steal operation successfully stole a chunk and therefore one of the first three conditions apply, therefore there is another operation that upon its completion a task will be returned.
\end{enumerate}
In the first two cases there is another consumer that returns a task and the claim holds. In the last three cases there is another consumer which either returns a task or is running an operation which, cannot end without a task being returned. If $c-1$ steals operation do not return a task, and during that time no other consumer returns a task, we know that all the consumers are running operations that cannot finish without a task being returned, either the $c$'th steal will not be interrupted and return a task, or one of the pending operations will return a task. Therefore, when this operation ends it is guaranteed that a task will be taken from the pool and the claim holds.
\end{proof}
%\subsection{Linearizability}
%\label{appendix:salsa-correctness}
%\begin{claim}
\label{salsa-insert-claim}
 If a SALSA operation returned a task, this task was previously inserted to a pool in the system by a producer.
\end{claim}
The proof of Claim~\ref{salsa-insert-claim} is immediate.



Before we proof the next cliam we proof the folowing lemma:

\begin{lemma}
 After the ownership of a chunk changes, the previous owner may only take one task, and that task will be pointer .
\end{lemma}
\begin{proof}
 
\end{proof}




\begin{claim}
\label{salsa-consumer-claim}
A task in a SALSA pool may be returned only once.
\end{claim}

\begin{proof}
First we note that a consumer returns a task only after changing the value of the task slot to TAKEN. 
Therefore, it is sufficient to show that only one consumer can change a slot to TAKEN.
Consumers change the value of a slot to taken in three places in the code (see Algorithm~\ref{alg:non-fifo}):
\begin{enumerate}
 \item Line~\ref{alg:line:cas-consumer} -- the consumer detected that the chunk was stolen and therefore performs a CAS operation to mark the slot as TAKEN.
 \item Line~\ref{alg:line:cas-steal} -- the consumer stole a chunk and now attempts to take a task using CAS operation.
 \item Line~\ref{alg:lines:fast-path} -- this is the consumer fast-path, the consumer tries to take a task without synchronization.
\end{enumerate}

When multiple consumers try to take a task using CAS operations, only one can succeed, therefore we only have to show that the third operation, which does not use CAS, cannot execute simultaneously with other operations.

For the operation on line~\ref{alg:lines:fast-path} to complete, the {\bf if} on line~\ref{alg:lines:consumer-owner-check} must be true. Therefore, we know that the current consumer was the owner of the chunk on line~\ref{alg:lines:ind-inc}.

\end{proof}

%\clearpage
%\section{Chunk size influence}
%\label{appendix:chunk-size}
%\begin{figure}[htb]
 \begin{center}
   \includegraphics[width=0.45\textwidth]{figures/chunk-size}
 \end{center}
	\caption{\footnotesize{System throughput as a function of the chunk size. }}
	\label{fig:chunk-size}
\end{figure}
Figure~\ref{fig:chunk-size} shows the influence of chunk size on system throughput for the chunk-based algorithms SALSA, SALSA+CAS and ConcBags in a $16/16$ workload. 
SALSA variations achieve their best throughput for large chunks with $1000$ tasks ($\sim8$KB size in $64$-bit architectures). The optimal chunk for ConcBags includes $128$ tasks. We believe that ConcBags is ineffective with large chunk sizes since its consumers linearly scan a chunk when seeking a task to steal. In contrast, SALSA keeps the index of the latest consumed task in the chunk node, and therefore its consume operations terminate in $O(1)$ steps for every chunk size. 
In our evaluation in section~\ref{sec:evaluation} we used optimal chunk sizes for each algorithm. 

%
%\end{appendix}

\end{document}