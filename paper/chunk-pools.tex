As described in Section~\ref{alg-structure}, each consumer keeps a pool of free chunks.
When a producer needs a new chunk for adding a task to consumer $c_i$, it tries to get a chunk from $c_i$'s chunk pool -- if no free chunks are available, the {\bf produce()} operation fails.

As described in Section~\ref{sec:system}, our system-wide policy defines that if an insertion operation fails, then the producer tries to insert a task to other pools. Thus, the producer avoids adding tasks to overloaded consumers, which in turn decreases the amount of costly steal operations. 

Another SALSA property is that a chunk is returned to the pool of a consumer that retrieves the latest task of this chunk. 
Therefore, the size of the chunk pool of consumer $c_i$ is proportional to the rate of $c_i$'s task consumption.
This property is especially appealing for heterogeneous systems -- a faster consumer $c_i$ ", (e.g., one running on a stronger or less loaded core), will have a larger chunk pool, and so more {\bf produce()} operations will insert tasks to $c_i$, automatically balancing the overall system load. 
