As described in Section~\ref{alg-structure}, each consumer has a pool of chunks of a limited size.
When a producer needs a new chunk to add to its list in consumer $c_i$, it tries to get a chunk from $c_i$'s chunk pool. If the pool doesn't contain any chunks, the the operation fails (if {\it produce}() was invoked) or it creates a new chunk (if {\it produceForce}() was invoked). As described in Section~\ref{sec:system}, our policy is first to invoke {\it produce}() on all the pools by order of distance from the producer, and only after the producer fails to insert the task to all pools it calls {\it produceForce}() on the closest pool. This policy reduces the number of steal operation by making producers balance the load, rather than the consumers. In addition, in SALSA a chunk is returned to the pool of the consumer which took the last task in that chunk. This balances the pool sizes, by making pools of overloaded consumers smaller and thus cause producers to move to other consumers.