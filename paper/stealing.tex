The stealing algorithm is given in the function {\bf steal()} in Algorithm~\ref{alg:non-fifo}. 
We refer to the stealing consumer as $c_s$, the victim process whose chunk is being stolen as $c_v$, and the stolen chunk as $ch$.

The idea is to turn $c_s$ to the exclusive owner of $ch$, such that $c_s$ will be able to take tasks from the chunk without synchronization. 
In order to do that, $c_s$ changes the ownership of $ch$ from $c_v$ to $c_s$ using CAS (line~\ref{alg:line:chown}) and removes the chunk from $c_v$'s list (line~\ref{alg:line:remove-chunk}). 
Once $c_v$ notices the change in the ownership it can take at most one more task from $ch$ (lines~\ref{alg:lines:stolen-chunk-begin}--\ref{alg:lines:stolen-chunk-end}). 

When the {\bf steal()} operation of $c_s$ occurs simultaneously with the {\bf takeTask()} operation of $c_v$, both $c_s$ and $c_v$ might try to retrieve the same task. We now explain why this might happen. Recall that $c_v$ notifies potential stealers of the task it is about to take by incrementing the \emph{idx} value in $ch$'s node (line~\ref{alg:lines:ind-inc}). This value is copied by $c_s$ in line~\ref{alg:line:copy-prev-node} when creating a copy of $ch$'s node for its steal list.

Consider, for example, a scenario in which the $idx$ is incremented by $c_v$ from $10$ to $11$. 
If $c_v$ checks $ch$'s ownership before it is changed by $c_s$, then $c_v$ takes the task at index $11$ \emph{without synchronization} (line~\ref{alg:lines:fast-path}). Therefore, $c_s$ cannot be allowed to take the task pointed by \emph{idx}. Hence, $c_v$ has to take the task at index $11$ even if it does observe the ownership change. 
After stealing the chunk, $c_s$ will eventually try to take the task pointed by $idx+1$. However, if $c_s$ copies the node before $idx$ is incremented by $c_v$, $c_s$ might think that the value of $idx+1$ is $11$. In this case, both $c_s$ and $c_v$ will try to retrieve the task at index $11$. To ensure that the task is not retrieved twice, both invoke CAS in order to retrieve this task (line~\ref{alg:line:cas-steal} for $c_s$, line~\ref{alg:line:cas-consumer} for $c_v$). 

The above algorithm works correctly as long as the stealing consumer can observe the node with the updated index value. 
This might not be the case if the same chunk is concurrently stolen by another consumer rendering the \emph{idx} of the original node obsolete. 
In order to prevent this situation, stealing a chunk from the pool of consumer $c_v$ is allowed only if $c_v$ is the owner of this chunk (line~\ref{alg:line:chown}). This approach is prone to the ABA problem: consider a scenario where consumer $c_a$ is trying to steal from $c_b$, but before the execution of the CAS in line~\ref{alg:line:chown}, the chunk is stolen by $c_c$ and then stolen back by $c_b$. In this case, $c_a$'s CAS succeeds but $c_a$ has an old value of $idx$. To prevent this ABA problem, the owner field contains a ``tag'', which is incremented on every CAS operation. For brevity, tags are omitted from the pseudo-code.

A na\"{i}ve way for $c_s$ to steal the chunk from $c_v$ would be first to change the ownership and then to move the chunk to the steal list. However, this approach may cause the chunk to ``disappear'' if $c_s$ is stalled, because the chunk becomes inaccessible via the lists of $c_s$ and yet $c_s$ is its owner. 
%In this case, as we explained above, the chunk cannot be stolen, which prevents other consumers from taking tasks from this chunk. 
Therefore, SALSA first adds the original node to the steal list of $c_s$, then change the ownership, and only then replaces the original node with a new one (lines~\ref{alg:line:resteal-begin}--\ref{alg:line:resteal-end}).