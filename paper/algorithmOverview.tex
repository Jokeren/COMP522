\subsubsection {SALSA producer operations}
\negspace
\begin{algo}[!ht]
\caption{SALSA implementation of SCPool: Producer Functions.}
\label{alg:producer-non-fifo}
\scriptsize
\begin{minipage}[t]{0.48\textwidth}
\begin{distribalgo}[1]
\setcounter{ALC@line}{\value{alg:non-fifo:lines}}

\INDENT {{\bf Producer local variables}:}
	\STATE int producerId
	\STATE Chunk chunk; initially $\bot$ \comment {the chunk to insert to} \label{alg:line:chunk}
	\STATE int prodIdx; initially $0$ \comment {the prefix of inserted tasks}
\ENDINDENT

\medskip

\INDENT {{\bf Function produce}(Task t, SCPool scPool):}
	\STATE {\bf return insert}(t, scPool, false)
\ENDINDENT

\medskip

\INDENT {{\bf Function insert}(Task t, SCPool scPool, bool force):}
	\INDENT {{\bf if} (chunk $= \bot$) {\bf then} \comment {allocate new chunk}}
		\STATE {\bf if} ({\bf getChunk}(scPool, force) $=$ {\bf false}) {\bf then return false}
	\ENDINDENT
	\STATE chunk.tasks[prodIdx] $\leftarrow$ t; prodIdx++ \label{alg:line:chunk-insert}
	\INDENT {{\bf if}(prodIdx $=$ CHUNK\_SIZE) {\bf then}}
	  \STATE chunk $\leftarrow \bot$ \comment {the chunk is full}
	\ENDINDENT
	\STATE {\bf return true}
\ENDINDENT

\setcounter{alg:non-fifo:lines}{\value{ALC@line}} % store the line number
\end{distribalgo}
\end{minipage}%
%
\hfill
%
\begin{minipage}[t]{0.48\textwidth}
%
\begin{distribalgo}[1]
\setcounter{ALC@line}{\value{alg:non-fifo:lines}}

\INDENT {{\bf Function produceForce}(Task t, SCPool scPool):}
	\STATE {\bf insert}(t, scPool, true)
\ENDINDENT

\medskip

\INDENT {{\bf Function getChunk}(SALSA scPool, bool force)}
	\STATE newChunk $\leftarrow$ dequeue chunk from scPool.chunkPool
	\INDENT {{\bf if} (chunk $= \bot$) \comment {no available chunks in this pool}} \label{alg:line:no-chunk-start}
		 \STATE {\bf if} (force $=$ false) {\bf then return false} 
		 \STATE newChunk $\leftarrow$ allocate a new chunk \label{alg:line:no-chunk-end}
	\ENDINDENT
	\STATE newChunk.owner $\leftarrow$ scPool.consumerId
	\STATE node $\leftarrow$ new node with idx $=-1$ and c $=$ newChunk
	\STATE scPool.chunkLists[producerId].{\bf append}(node)
	\STATE chunk $\leftarrow$ newChunk; prodIdx $\leftarrow 0$ 
	\STATE {\bf return true}
\ENDINDENT

\setcounter{alg:non-fifo:lines}{\value{ALC@line}}
\end{distribalgo}
\end{minipage}
\end{algo}

The description of SALSA producer functions is depicted in Algorithm~\ref{alg:producer-non-fifo}. 
Generally, the insertion of a new task by a producer consists of two stages: 
1) finding a chunk to insert a task to, and 2) inserting a task to the chunk. 

\paragraph {Finding a chunk.}
The local producer variable \emph{chunk} keeps a current chunk to insert the tasks to (line~\ref{} in Algorithm~\ref{}). 
When a producer starts working on a chunk $c$, the \emph{chunk} variable points to $c$ and the producer continues inserting the tasks to $c$ until it fills all the chunk entries. 
Note that according to this approach a producer is oblivious to chunk stealing -- it is not aware of the fact that the chunk might be moved from one pool to another.  
If the \emph{chunk}'s value is $\bot$, it means that the previous chunk has been filled up. In this case a producer should get a new chunk (Function \emph{getChunk} in Algorithm~\ref{}). 
To this end, the producer tries to get a chunk from the chunk pool, which is implemented as a FIFO non-blocking queue, and append it to the appropriate chunk list. If the chunk pool is empty then the producer either allocates a new chunk by itself (\emph{produceForce()} function), or fails to insert the task and return $\bot$ (\emph{produce()} function) (lines~\ref{}--\ref{} in Algorithm~\ref{}). 

\paragraph {Inserting a task to the chunk.}
As previously described in Section~\ref{alg-structure}, different producers insert tasks to different chunk lists, which essentially removes the need for synchronization among producers. 
The local producer variable \emph{prodIdx} indicates the insertion entry in the chunk's task array.
All that the insertion function has to do is to put a task in the entry pointed by the \emph{prodIdx} and to increment this index.
If the index arrives to the maximal value, \emph{chunk} variable is set to $\bot$, indicating that the next insertion operation should start a new chunk. 