\subsubsection {SALSA producer algorithm}
\begin{algo}[!ht]
\caption{SALSA implementation of SCPool: Producer Functions.}
\label{alg:producer-non-fifo}
\scriptsize
\begin{minipage}[t]{0.48\textwidth}
\begin{distribalgo}[1]
\setcounter{ALC@line}{\value{alg:non-fifo:lines}}

\INDENT {{\bf Producer local variables}:}
	\STATE int producerId
	\STATE Chunk chunk; initially $\bot$ \comment {the chunk to insert to}
	\STATE int prodIdx; initially $0$ \comment {the prefix of inserted tasks}
\ENDINDENT

\medskip

\INDENT {{\bf Function produce}(Task t, SALSA scPool):}
	\STATE {\bf return insert}(t, scPool, false)
\ENDINDENT

\medskip

\INDENT {{\bf Function insert}(Task t, SCPool scPool, bool force):}
	\INDENT {{\bf if} (chunk $= \bot$) {\bf then}}
		\STATE {\bf if} ({\bf getChunk}(scPool, force) $=$ {\bf FULL}) {\bf then return FULL}
	\ENDINDENT
	\STATE chunk.tasks[prodIdx] $\leftarrow$ t; prodIdx++
	\INDENT {{\bf if}(prodIdx $=$ CHUNK\_SIZE) {\bf then}}
	  \STATE chunk $\leftarrow \bot$ \comment {the chunk is full}
	\ENDINDENT
	\STATE {\bf return SUCCESS}
\ENDINDENT

\setcounter{alg:non-fifo:lines}{\value{ALC@line}} % store the line number
\end{distribalgo}
\end{minipage}%
%
\hfill
%
\begin{minipage}[t]{0.48\textwidth}
%
\begin{distribalgo}[1]
\setcounter{ALC@line}{\value{alg:non-fifo:lines}}

\INDENT {{\bf Function produceForce}(Task t, SALSA scPool):}
	\STATE {\bf return insert}(t, scPool, true)
\ENDINDENT

\medskip

\INDENT {{\bf Function getChunk}(SALSA scPool, bool force)}
	\STATE newChunk $\leftarrow$ a chunk from scPool.chunkPool
	\INDENT {{\bf if} (chunk $= \bot$) \comment {no available chunks in this pool}}
		 \STATE {\bf if}(force $=$ false) {\bf then return FULL} 
		 \STATE newChunk $\leftarrow$ allocate a new chunk
	\ENDINDENT
	\STATE newChunk.owner $\leftarrow$ scPool.consumerId
	\STATE node $\leftarrow$ new node with idx $=-1$ and c $=$ newChunk
	\STATE scPool.chunkLists[producerId].{\bf append}(node)
	\STATE chunk $\leftarrow$ newChunk; prodIdx $\leftarrow 0$ 
	\STATE {\bf return SUCCESS}
\ENDINDENT

\setcounter{alg:non-fifo:lines}{\value{ALC@line}}
\end{distribalgo}
\end{minipage}
\end{algo}

The description of SALSA producer functions is presented in Algorithm~\ref{alg:producer-non-fifo}. 
The insertion of a new task by a producer consists of two stages: 
1) finding a chunk to insert a task to, and 2) inserting a task to the chunk. 

\paragraph {Finding a chunk.}
The producer local variable \emph{chunk} keeps the chunk to insert the tasks to (line~\ref{alg:line:chunk} in Algorithm~\ref{alg:producer-non-fifo}). 
When a producer starts working on a chunk $c$ it continues inserting the tasks to $c$ until it is full. 
Note that the producer is oblivious to the fact that the chunk might be moved from one pool to another.
If the \emph{chunk}'s value is $\bot$, it means that the previous chunk has been filled up, and the producer should get a new chunk (Function \emph{getChunk}). 
In this case, the producer tries to get a chunk from the chunk pool and append it to the appropriate chunk list. If the chunk pool is empty then the producer either allocates a new chunk by itself (\emph{produceForce()} function), or returns $\bot$ (\emph{produce()} function) (lines~\ref{alg:line:no-chunk-start}--\ref{alg:line:no-chunk-end}). 

\paragraph {Inserting a task to the chunk.}
As previously described in Section~\ref{alg-structure}, different producers insert tasks to different chunks, which removes the need for synchronization among producers. 
The producer local variable \emph{prodIdx} indicates the next free slot in the chunk.
All that the insertion function has to do is to put a task in that slot and to increment \emph{prodIdx} (line~\ref{alg:line:chunk-insert}).
Once the index reaches the maximal value, the \emph{chunk} variable is set to $\bot$, indicating that the next insertion operation should start a new chunk. 

\subsubsection {SALSA consumer fast-path}
\begin{algo}[!ht]
\caption{SALSA implementation of SCPool: Consumer Functions.} 
\label{alg:non-fifo}
\scriptsize
\begin{minipage}[t]{0.48\textwidth}
\begin{distribalgo}[1]
\setcounter{ALC@line}{\value{alg:non-fifo:lines}}
\smallskip


\INDENT {{\bf Function consume}():}
  \INDENT {{\bf if} (currentNode $\neq \bot$) {\bf then} \comment {common case}}
		\STATE t $\leftarrow$ {\bf takeTask}(currentNode)
		\STATE {\bf if} (t $\neq \bot$) {\bf then return} t
	\ENDINDENT
	\INDENT {{\bf foreach} Node $n$ in ChunkLists {\bf do:} \comment {fair traversal of chunkLists}}
	  \INDENT {{\bf if} (n.c $\neq \bot \wedge \textrm{n.c.owner} = \textrm{consumerId}$) {\bf then}}
  			\STATE t $\leftarrow$ {\bf takeTask}(n)
				\STATE {\bf if} (t $\neq \bot$) {\bf then} currentNode $\leftarrow$ n; {\bf return} t
			\ENDINDENT
	\ENDINDENT
% 	\comment {Iterate over other chunk lists}
% 	\INDENT {{\bf foreach} cl {\bf in} chunkLists {\bf do} \comment {fair traversal}} 
%   	\INDENT {{\bf foreach} node {\bf in} cl {\bf do}}
%   	  \INDENT {{\bf if}(node.c $\neq \bot \wedge \textrm{node.c.owner} = \textrm{consumerId}$) {\bf then}}
%   			\STATE t $\leftarrow$ {\bf takeTask}(node)
% 				\STATE {\bf if} (t $\neq \bot$) {\bf then} currentNode $\leftarrow$ node; {\bf return} t
% 			\ENDINDENT
%   	\ENDINDENT
% 	\ENDINDENT
	\STATE currentNode $\leftarrow \bot$; {\bf return} $\bot$
\ENDINDENT

\medskip

\INDENT {{\bf Function takeTask}(Node n):}
  \STATE chunk $\leftarrow$ n.c
  \STATE {{\bf if} (chunk $= \bot$) {\bf then return $\bot$} \comment{this chunk has been stolen}}
 
  \STATE task $\leftarrow$ chunk.tasks[n.idx + 1]
  \STATE {\bf if} (task $= \bot$) {\bf then return} $\bot$ \comment{no inserted tasks}
 	
 	\smallskip 
  \STATE n.idx++ \comment {tell the world you're going to take a task from idx} \label{alg:lines:ind-inc}
  \INDENT {{\bf if} (chunk.owner $=$ consumerId) {\bf then} \comment {common case}}
 		\STATE chunk.tasks[n.idx] $\leftarrow$ TAKEN \label{alg:lines:fast-path}
  	\STATE {\bf checkEmpty}(n)
  	\STATE {\bf return} task 
  \ENDINDENT
  
  \smallskip
  \comment {the chunk has been stolen, CAS the last task and go away} 
 	\STATE success $\leftarrow$ (task $\neq$ TAKEN $\wedge$ \\ \label{alg:lines:stolen-chunk-begin}
 		\hspace{0.5cm} CAS(chunk.tasks[n.idx], task, TAKEN)) \label{alg:line:cas-consumer}
 	\STATE {\bf if}(success) {\bf then checkEmpty}(n) 
	\STATE currentNode $\leftarrow \bot$
 	\STATE {\bf return} (success) ? task : $\bot$ \label{alg:lines:stolen-chunk-end}
\ENDINDENT



\medskip

\INDENT {{\bf Function checkEmpty}(Node n):}
	\INDENT {{\bf if}(n.idx + 1 $=$ CHUNK\_SIZE) {\bf then} \comment {finished the chunk}}
  	\STATE n.c $\leftarrow \bot$; return chunk to chunkPool
  	\STATE currentNode $\leftarrow \bot$
  \ENDINDENT
\ENDINDENT


\setcounter{alg:non-fifo:lines}{\value{ALC@line}} % store the line number
\end{distribalgo}
\end{minipage}%
%
\hfill
%
\begin{minipage}[t]{0.48\textwidth}
%
\begin{distribalgo}[1]
\setcounter{ALC@line}{\value{alg:non-fifo:lines}}
\smallskip

\INDENT {{\bf Function steal}(SCPool from):}
	\STATE prevNode $\leftarrow$ a node holding tasks from some list \comment {different policies possible} \label{alg:line:take-steal-chunk}
	\STATE c $\leftarrow$ prevNode.c; {\bf if} (c = $\bot$) {\bf then return} $\bot$

	\STATE chunkLists[steal].{\bf append}(prevNode) \comment {make it restealable} \label{alg:line:resteal-begin}
	\INDENT {{\bf if} ({\bf CAS}(c.owner, from.consumerId, consumerId) $=$ false)} \label{alg:line:chown}
 		\STATE chunkLists[steal].{\bf remove}(prevNode)
 		\STATE {\bf return} $\bot$ \comment {failed to steal}
	\ENDINDENT

	\smallskip
	\STATE newNode $\leftarrow$ copy of prevNode \label{alg:line:copy-prev-node}
	\INDENT {{\bf if} (newNode.idx+1 $=$ CHUNK\_SIZE) \comment {Chunk is empty}} \label{alg:line:steal-node-empty}
	  \STATE chunkLists[steal].{\bf remove}(prevNode)
	  \STATE {\bf return} $\bot$
	\ENDINDENT
	\STATE replace prevNode with newNode in chunkLists[steal]
	\STATE prevNode.c $\leftarrow \bot$ \comment {remove chunks from consumer's list} \label{alg:line:remove-chunk} \label{alg:line:resteal-end}
	
	\smallskip
	\comment {done stealing the chunk, take one task from it}
	\STATE idx $\leftarrow$ newNode.idx
	\STATE task $\leftarrow$ c.tasks[idx+1] 
	\STATE {\bf if} (task $= \bot$) {\bf then return} $\bot$ \comment {still no task at idx+1} \label{alg:line:steal-chunk-not-full}
	\INDENT {{\bf if} (task $=$ TAKEN $\vee$ \\
		\hspace{0.5cm} !{\bf CAS}(c.tasks[idx+1], task, TAKEN)) {\bf then}} \label{alg:line:cas-steal} 
		\STATE task $\leftarrow \bot$
	\ENDINDENT
	\STATE currentNode $\leftarrow$ newNode
	\STATE {\bf if} (task $\neq \bot$) {\bf then checkEmpty}(newNode)
	\STATE newNode.idx $\leftarrow$ newNode.idx+1
	\STATE {\bf return} task
\ENDINDENT

\setcounter{alg:non-fifo:lines}{\value{ALC@line}}
\end{distribalgo}
\end{minipage}
\end{algo}


We now describe the way the consumer retrieves a task from the chunk in the common case, when no simultaneous stealing. 
Unlike producers that operate solely on the chunks, a consumer should take into consideration the possibility of stealing. 
Therefore, it should notify other processes about the task it is about to take. 

Each node in the chunk list keeps an index of the taken prefix of its chunk (\emph{idx} variable, which is initiated to $-1$). 
A consumer that wants to take a task $T$ first increments the index, checks the ownership and then changes the chunk entry from $T$ to a special value \emph{TAKEN}. According to this algorithm, \emph{idx} always points to the last taken task or to a task that is about to be taken. 
This way, a process that is stealing a chunk from the node with $\textit{idx} = i$ can assume that the tasks in the range $[0 \ldots i]$ have already been taken.



