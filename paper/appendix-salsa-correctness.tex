\begin{claim}
\label{salsa-insert-claim}
 If a SALSA operation returned a task, this task was previously inserted to a pool in the system by a producer.
\end{claim}
The proof of Claim~\ref{salsa-insert-claim} is immediate.



Before we proof the next cliam we proof the folowing lemma:

\begin{lemma}
 After the ownership of a chunk changes, the previous owner may only take one task, and that task will be pointer .
\end{lemma}
\begin{proof}
 
\end{proof}




\begin{claim}
\label{salsa-consumer-claim}
A task in a SALSA pool may be returned only once.
\end{claim}

\begin{proof}
First we note that a consumer returns a task only after changing the value of the task slot to TAKEN. 
Therefore, it is sufficient to show that only one consumer can change a slot to TAKEN.
Consumers change the value of a slot to taken in three places in the code (see Algorithm~\ref{alg:non-fifo}):
\begin{enumerate}
 \item Line~\ref{alg:line:cas-consumer} -- the consumer detected that the chunk was stolen and therefore performs a CAS operation to mark the slot as TAKEN.
 \item Line~\ref{alg:line:cas-steal} -- the consumer stole a chunk and now attempts to take a task using CAS operation.
 \item Line~\ref{alg:lines:fast-path} -- this is the consumer fast-path, the consumer tries to take a task without synchronization.
\end{enumerate}

When multiple consumers try to take a task using CAS operations, only one can succeed, therefore we only have to show that the third operation, which does not use CAS, cannot execute simultaneously with other operations.

For the operation on line~\ref{alg:lines:fast-path} to complete, the {\bf if} on line~\ref{alg:lines:consumer-owner-check} must be true. Therefore, we know that the current consumer was the owner of the chunk on line~\ref{alg:lines:ind-inc}.

\end{proof}
