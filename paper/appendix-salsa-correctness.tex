\begin{claim}
\label{salsa-insert-claim}
 If a SALSA operation returns a task, this task has been previously inserted to a pool by a producer.
\end{claim}
\noindent
The proof of Claim~\ref{salsa-insert-claim} is immediate.

\vspace{5pt}\noindent
%Before we prove the next claim we prove the following lemmas:
\begin{lemma}
 \label{lemma:steal-take-bound}
 Let $i$ be the $idx$ of the node used by a consumer that is about to execute line~\ref{alg:lines:fast-path}. Then consumers that steal the chunk pointed by that node cannot take tasks from slots smaller than $i+1$. 
\end{lemma}
\begin{proof}
 If a consumer $c$ reached line~\ref{alg:lines:fast-path} then since the {\bf if} statement in line~\ref{alg:lines:consumer-owner-check} was true, we conclude that if a consumer stole the chunk it saw the value $i$ that $c$ uses in line~\ref{alg:lines:fast-path} because the stealer reads this value \emph{after} changing ownership. Since the stealer only take tasks from slots greater than the value it read, we deduce that it only take tasks from slots greater than $i$.
\end{proof}

\begin{lemma}
 \label{lemma:no-inc-after-steal}
 Let $i$ be the $idx$ of a node as read by a stealing consumer after changing chunk ownership in line~\ref{alg:line:chown}. Then the previous owner of the chunk cannot take tasks in slots with index greater than $i+1$ as long as the ownership does not change.
\end{lemma}
\begin{proof}
Once a consumer notices that its chunk was stolen it sets $currentNode$ to $\bot$ (line~\ref{alg:line:check-owner-stealer}) or leaves it as $\bot$ (line~\ref{alg:lines:stolen-chunk-end}). If a consumer takes a Node from its pool, it makes sure that it is the owner of this Node's chunk, i.e., when a consumer notices that a chunk has been stolen it stops using the node that pointed to it.

If a consumer increases the $idx$ of a Node, it checks if it is still its owner (lines~\ref{alg:lines:consumer-owner-check} and~\ref{alg:line:check-owner-stealer}). Therefore, once the ownership of a chunk changes, the $idx$ of a Node that pointed to that chunk before it was stolen, can increase by at most 1. Hence, the value of $i$ is at least $idx-1$.

When a consumer takes a task in {\bf consume()} (lines~\ref{alg:lines:fast-path} and~\ref{alg:line:cas-consumer}), it takes it from slot $k$, where $k$ is the current $idx$ of the current consumer's Node. Since we know that after a {\bf steal()} operation the $idx$ value $i$ that is read by the stealer is at least $k-1$, we conclude that the previous owner of the chunk can take tasks only from slots less than $i+1$, unless it steals this chunk again.

When a consumer takes a task in {\bf steal()}, it takes it from slot $k+1$ and then increments $idx$. Therefore, when another consumer steals this chunk, the value $i$ that it reads~\ref{alg:line:copy-prev-node} is at least $k$. As shown earlier, the original owner will not take a task greater than $k+1$ without stealing this chunk again. 
\end{proof}

\begin{claim}
\label{salsa-consumer-claim}
A task in a SALSA pool may be only returned once.
\end{claim}

\begin{proof}
A consumer returns a task only after it changes the value of the task slot to TAKEN. 
Therefore, it is sufficient to show that only one consumer can change a given slot to TAKEN.
There are three places in the code a value can be changed to TAKEN (see Algorithm~\ref{alg:non-fifo}):
\begin{enumerate}
 \item Line~\ref{alg:line:cas-consumer} -- a consumer detected that the chunk has been stolen and therefore performs a CAS operation to mark the slot as TAKEN.
 \item Line~\ref{alg:line:cas-steal} -- a consumer attempts to take a task using CAS operation after it steals a chunk.
 \item Line~\ref{alg:lines:fast-path} -- a consumer tries to take a task without synchronization (fast-path).
\end{enumerate}

When multiple consumers try to take a task using CAS operations, only one can succeed, therefore, it is enough to show that the operation that does not use CAS, cannot compete with other operations that try to take the same task.

We now show that if the operation in line~\ref{alg:lines:fast-path} takes a task, no other operation takes this task. 
Consider a consumer $a$ that is about to execute line~\ref{alg:lines:fast-path}.
According to Lemma~\ref{lemma:steal-take-bound}, consumers that steal the chunk used by $a$, cannot take the task that $a$ is about to take.

After stealing a Node, the consumer increases $idx$ by 1 (see line~\ref{alg:line:steal-idx-inc}), and therefore the next time it invokes {\bf consume()} it will take the task from $i+2$ where $i$ is the value it read from the previous Node in line~\ref{alg:line:copy-prev-node}. Therefore, in case $a$ stole from another consumer $b$, when $a$ executes line~\ref{alg:line:copy-prev-node}, then $b$ can only take tasks from slots smaller than $i+2$. Hence, $b$ cannot take the task that $a$ retrieves in line~\ref{alg:lines:fast-path} unless it resteals the chunk. 
\end{proof}
